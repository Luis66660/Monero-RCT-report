% this file is called up by main.tex
% content in this file will be fed into the main document

% ---------------------------------------------------------------------------

\chapter{Introduction}
\label{chap:introduction}

The purpose of blockchains is to furnish trust for operations between unrelated parties, without requiring the collaboration of any trusted third party.
\\ \newline
Trust is attained through the use of cryptographic artifacts, which allow data registered in an easily accessible database – the blockchain - to be virtually immutable and non-falsifiable. In other words, a blockchain is a public, distributed database containing data whose legitimacy cannot be disputed by any party.
\\ \newline
Cryptocurrencies store transactions in the blockchain, which acts as a public ledger of all the verified currency operations. Most cryptocurrencies store transactions in clear text, to facilitate verification of transactions by the community.
\\ \newline
Clearly, an open blockchain defies any basic understanding of privacy, since it virtually {\em publicizes} the complete transaction histories of its users.
\\ \newline
To address the lack of privacy, users of cryptocurrencies such as Bitcoin can obfuscate transactions by using temporary intermediate addresses \cite{DBLP:journals/corr/NarayananM17}. However, with appropriate tools it is possible to analyze flows and to a large extent link true senders with receivers \cite{DBLP:journals/corr/ShenTuY15b, DK-police-tracing-btc, Andrew-Cox-Sandia}.

In contrast, the cryptocurrency Monero attempts to tackle the issue of privacy by storing only stealth, single-use addresses for receipt of funds in the blockchain, and by authenticating the dispersal of funds in each transaction with ring signatures. With these methods there are no effective ways to link senders with receivers or trace the origin of funds \cite{Monero-intro}.

Additionally, transaction amounts in the Monero blockchain are concealed behind cryptographic constructions, rendering currency flows opaque.

The result is a cryptocurrency with a high level of privacy.




\section{Objectives}
\label{sec:goals}

Monero is a cryptocurrency of recent creation, yet it displays a steady growth in popularity\footnote{\label{marketcap_note}As of December 28\nth, 2017, Monero occupies the 10\nth position as regards market capitalization, see\\ \url{https://coinmarketcap.com/}}. 
Unfortunately, there is little comprehensive documentation describing the mechanisms it uses. Even worse, important parts of its theoretical framework have been published in non peer-reviewed papers which are incomplete and/or contain errors. For significant parts of the theoretical framework of Monero, only the source code is reliable as a source of information.
  
We intend to palliate this situation by collecting in-depth information about Monero’s inner workings, reviewing algorithms and cryptographic schemes, and discussing the degree to which they might afford sufficient transaction privacy and security to its users.
\\ 
\\ 

We have centered our attention on release 0.11.1.0 of the Monero software suite, the most recent release at the moment this is written. All transaction related mechanisms described here belong to this version. Deprecated transaction schemes have not been explored to any extent, even if they may be partially supported for backward compatibility reasons.


\section{Readership}

We expect the reader to possess a basic understanding of discrete mathematics and algebraic structures, but possibly only fundamental insights in the field of cryptography. We also expect the user to have a basic understanding of how a cryptocurrency like Bitcoin works. For technically oriented laymen we have tried to fill potential knowledge gaps in the footnotes.

A reader with this background should be able to follow our constructive, step-by-step description of the elements of the Monero cryptocurrency.

We have purposefully omitted, or delegated to footnotes, some mathematical technicalities, when they would be in the way of clarity. We have also omitted concrete implementation details where we thought they were not essential. Our objective has been to present the subject half-way between mathematical cryptography and computer programming, aiming at completeness and conceptual clarity.



\section{Origins of the Monero cryptocurrency}

The cryptocurrency Monero, originally known as BitMonero, was created in April, 2014 as a derivative of the proof-of-concept currency CryptoNote.

CryptoNote is a cryptocurrency devised by various individuals. A landmark whitepaper describing it was published under the pseudonym of Nicolas van Saberhagen in October 2013 \cite{cryptoNoteWhitePaper}. It offered sender and receiver anonymity through the use of one-time addresses, and untraceability of flows by means of ring signatures.

Since its inception, Monero has further strengthened its privacy aspects by implementing amount hiding, as described by Greg Maxwell (among others) in \cite{Signatures2015BorromeanRS}, as well as Shen Noether's improvements
on ring signatures \cite{ledger34}.
  

\section{Outline}

As hinted earlier, our aim is to deliver a self-contained and step-by-step description of the Monero cryptocurrency. This report has been structured to fulfill this objective, leading the reader through all elements needed to describe the currency’s inner workings.
\\

In our quest for comprehensiveness, we have chosen to present all the basic elements of cryptography needed to understand the complexities of Monero. In Chapter \ref{chap:basicConcepts} we develop essential aspects of Elliptic Curve cryptography.

Chapter \ref{chapter:ring-signatures} outlines the ring signature related algorithms that will be applied to achieve confidential transactions while preventing double-spending attacks.

In Chapter \ref{chapter:pedersen-commitments} we introduce the cryptographic mechanisms used to conceal amounts.

Finally, with all the components in place, we will be able to expose the transaction schemes used in Monero in Chapter \ref{chapter:transactions}.


%While seemingly outside the main focus of this thesis, we also found that
%the consensus algorithm in Monero would interest readers. 
%We feel that it is a valuable contribution in the cryptocurrency world,
%as it brings consensus  power back to the plain participants in the network. In doing so it ameliorates security
%and resilience to attacks relying on computing power. We have dedicated Chapter  \ref{chapter:consensus}
%to putting forth the principles of the consensus algorithm in Monero.


Appendices \ref{appendix:RCTTypeFull} and \ref{appendix:RCTTypeSimple} describe the structure of sample transactions in the blockchain, providing a connection between the theoretical elements described in earlier sections with their real-life implementation.