\chapter{Monero in an Escrowed Marketplace}
\label{chapter:escrowed-market}

Most of the time purchasing from an online retailer goes smoothly. A buyer sends money to a vendor, and the expected product arrives at their doorstep. If the product has a defect, or in some cases if the buyer changes their mind, they can return it and get a refund.

It's difficult to trust a person or organization you have never met, and many shoppers can feel safe in their purchases knowing their credit card company will reverse a payment on request \cite{credit-card-reversals}.

Cryptocurrency transactions are not reversible, and there is limited legal recourse a buyer or seller can take when something goes wrong, especially for Monero which is not open to easy chain analysis \cite{chainalysis-2020-report}. The cornerstone of robust online shopping with cryptocurrencies is 2-of-3 multisignature-based escrowed exchanges, which enable third parties to mediate disputes. By trusting those third parties, even completely anonymous vendors and shoppers can interact without formality.

Since Monero multisig interactions can be rather complex (recall Section \ref{sec:simplified-communication}), we dedicate this chapter to describing a maximally efficient escrowed shopping environment.\footnote{Ren\'e ``rbrunner7" Brunner, a Monero contributor who created the MMS \cite{mms-project-proposal, mms-manual}, investigated integrating Monero multisig into the decentralized cryptocurrency-based digital marketplace OpenBazaar \url{https://openbazaar.org/}. The concepts laid out here are inspired by the hurdles Ren\'e encountered \cite{openbazaar-rbrunner-investigation} (his `earlier analysis').}\footnote{Our initial impression is Monero's current implementation of multisig already has a similar transaction creation flow to what we need for an escrowed shopping environment, which is good news for potential implementation efforts. Readers should note that Monero multisig requires some security updates before it can see heavy use \cite{multisig-research-issue-67}.}



\section{Essential features}
\label{sec:escrowed-marketplace-essential-features}

There are several basic requirements and features for streamlined interactions between buyers and sellers online. We take these points from Ren\'e's OpenBazaar investigation  \cite{openbazaar-rbrunner-investigation}, as they are quite sensible and extensible.
\begin{itemize}
    \item {\em Offline selling}: A buyer should be able to access a vendor's online shop and place an order while the vendor is offline. Obviously, the vendor must actually go online to receive the order and fulfill it.\footnote{Fulfilling a purchase order means sending the product out to be delivered to the purchaser.}
    \item {\em Purchase order-based payments}: Vendor addresses for receipt of funds are unique for each purchase order, in order to match orders and payments.
    \item {\em High-trust purchasing}: A buyer may, if they trust a vendor, pay for a product even before it has been fulfilled or delivered.
    \begin{itemize}
        \item {\em Online direct payment}: After confirming the vendor is online, and that their product listing is available, the purchaser sends money in a single transaction to the vendor via a vendor-provided address, who then signals the order is being fulfilled.
        \item {\em Offline payment}: If a vendor is offline, the buyer creates and funds a 1-of-2 multisig address with enough money to cover their intended purchase. When the vendor comes online, they may take money out of the multisig address (sending change back to the buyer), and fulfill the order. If the vendor never comes back online (or e.g. after some reasonable waiting period), or if the buyer changes their mind before he comes back, the buyer may empty the 1-of-2 address back into her personal wallet.
    \end{itemize}{}
    \item {\em Moderated purchasing}: A 2-of-3 multisig address is constructed between the buyer, vendor, and a moderator agreed on by both buyer and vendor. The buyer funds this address before vendor fulfillment, and then once the product is delivered two of the parties cooperate to release the funds. If the vendor and buyer can't reach an agreement, they may enlist the moderator's judgement.
\end{itemize}

We will not cover high-trust purchasing, as, without any complex communication required, the features are fairly trivial.\footnote{1-of-2 multisig can take advantage of some concepts useful for 2-of-3 multisig, in particular constructing the address in the first place.}


\subsection{Purchasing workflow}
\label{subsec:escrowed-marketplace-purchasing-workflow}

All purchases should fit within the same set of steps, assuming all parties do their due diligence. A number of steps are related to what a moderator should expect when stepping in, e.g. ``did you request a refund, before getting me involved?".
\begin{enumerate}
    \item A buyer accesses the vendor's online shop, identifies a purchase to make, selects `I want to purchase it', makes funding available for that purchase, then submits the purchase order to the vendor.
    \item The vendor receives the purchase order, verifies the product is in stock and the available funding is enough, and either returns money to the buyer or fulfills the order by shipping out the product and notifying the buyer with a receipt.
    \begin{itemize}
        \item For a 2-of-3 multisig, the buyer can optionally authorize payment when receiving notification of fulfillment.
    \end{itemize}{}
    \item The buyer either receives the product as expected, or doesn't receive the product on time or receives a defective product.
    \begin{itemize}
        \item {\em Good product}: The buyer either gives feedback to the vendor, or does nothing.
        \begin{itemize}
            \item {\em Buyer sends feedback}: The buyer can leave either positive, or negative feedback.
            \begin{itemize}
                \item {\em Positive feedback}: If this is a 2-of-3 multisig that hasn't been paid yet, then this is the step where the buyer confirms payment to the vendor. Otherwise it's just a positive review. [END OF WORKFLOW]
                \item {\em Negative feedback}: If this is a 2-of-3 multisig, then this step leads into the `bad product' workflow. Otherwise it's just a negative review.
            \end{itemize}{}
            \item {\em Buyer does nothing}: Either the vendor has already been paid, or it's a 2-of-3 multisig and he needs to cooperate with someone to receive funds.
            \begin{itemize}
                \item {\em Vendor has been paid}: [END OF WORKFLOW]
                \item {\em Vendor has not been paid}: The vendor pursues payment.
                \begin{enumerate}
                    \item The vendor contacts the buyer requesting payment (or sends out a reminder).
                    \item The buyer either responds, or does not respond.
                    \begin{itemize}
                        \item {\em Buyer responds}: The buyer's response can be to make a payment, or pursue a refund.\\
                        $>$ {\em Buyer makes payment}: [END OF WORKFLOW]\\
                        $>$ {\em Buyer pursues refund}: Go to the `bad product' workflow.
                        \item {\em Buyer does not respond}: The vendor enlists the moderator's help to release funds. [END OF WORKFLOW]
                    \end{itemize}{}
                \end{enumerate}{}
            \end{itemize}
        \end{itemize}{}
        \item {\em No product or bad product}: The buyer pursues a refund.
        \begin{enumerate}
            \item The buyer contacts the vendor requesting a refund, perhaps with an explanation.
            \item The vendor complies with the refund request, or contests it (or ignores it).
            \begin{itemize}
                \item {\em Vendor complies}: Money is refunded to the buyer. [END OF WORKFLOW]
                \item {\em Vendor contests}: Either it's not a 2-of-3 multisig, or it is.
                \begin{itemize}
                    \item {\em It's not a 2-of-3}: [END OF WORKFLOW]
                    \item {\em It's a 2-of-3}: Either the buyer gives up on the refund request (literally, or by failing to respond in a timely manner), or the buyer is adamant.
                    \begin{itemize}
                        \item {\em Buyer gives up}: Either the buyer authorized the vendor's payment, or not.\\
                        $>$ {\em Buyer authorized payment}: [END OF WORKFLOW]\\
                        $>$ {\em Buyer didn't authorize}: The vendor contacts the moderator, who authorizes the payment. [END OF WORKFLOW]
                        \item {\em Buyer is adamant}: The vendor or buyer contacts the moderator, who cooperates with the participants to pass judgement. [END OF WORKFLOW]
                    \end{itemize}{}
                \end{itemize}
            \end{itemize}{}
        \end{enumerate}{}
    \end{itemize}{}
\end{enumerate}{}



\section{Seamless Monero multisig}
\label{sec:escrowed-marketplace-seamless-multisig}

We can take advantage of the natural purchase order workflow to squeeze in nearly all parts of a Monero 2-of-3 multisignature interaction without the participants even noticing. There is an extra small step for the vendor at the end, where he must sign and submit the final transaction to receive payment, analogous to `emptying the cash register'.\footnote{Extending this beyond (N-1)-of-N is likely infeasible without more steps, due to the additional rounds necessary for setting up a lower threshold multisig address.}


\subsection{Basics of a multisig interaction}
\label{subsec:escrowed-marketplace-multisig-interaction-basics}

All 2-of-3 multisig interactions contain the same set of communication rounds, involving address setup and transaction building. In order to comply with the normal workflow we use a reorganized transaction construction process compared to our multisig chapter (recall Sections \ref{sec:simplified-communication} and \ref{sec:n-1-of-n}).\footnote{This procedure is actually quite similar to how Monero currently organizes multisig transactions.}
\begin{enumerate}
    \item {\em Address setup}
    \begin{enumerate}
        \item All users must first learn the other participants' base keys, which they will use to construct shared secrets. They transmit the public keys of those shared secrets (e.g. $K^{sh} = \mathcal{H}_n(k^{base}_A*k^{base}_B G) G$) to the other users.
        \item After learning all shared secret public keys, each user may {\tt premerge} and then {\tt merge} them into the address's public spend key. The aggregation private spend keys will be used for signing transactions. A hash of the primary signers' (the buyer and vendor) shared secret private key (e.g. $k^{sh} = \mathcal{H}_n(k^{base}_A*k^{base}_B G)$) will be used as the view key (e.g. $k^v = \mathcal{H}_n(k^{sh})$). We go into more detail on these keys later (Section \ref{subsec:escrowed-marketplace-escrow-user-experience}).
    \end{enumerate}{}
    \item {\em Transaction building}: Assume the address owns at least one output already, and the key images are unknown. There are two signers, the initiator and the cosigner.
    \begin{enumerate}
        \item {\em Initiate a transaction}: The initiator decides to start a transaction. She generates opening values (e.g. $\alpha G$) for all owned outputs (she isn't sure which ones will get used yet), and commits to them. She also creates partial key images for those owned outputs, and signs them with a proof of legitimacy (recall Section \ref{sec:recalculating-key-images-multisig}). She sends that information, along with her own personal address for receipt of funds (e.g. for change if appropriate, etc.), to the cosigner.
        \item {\em Make a partial transaction}: The cosigner verifies that all information in the initiated transaction is valid. He decides the output recipients and amounts (they could be partially based on the initiator's recommendation), inputs to be used along with their respective ring member decoys, and the transaction fee. He generates a transaction private key (or keys if there are subaddresses involved), creates one-time addresses, output commitments, encoded amounts, pseudo output commitment masks, and opening values for the commitments to zero. To prove amounts are in range, he builds the Bulletproof range proof for all outputs. He also generates opening values for his signature (but does not commit to them), random scalars for the MLSAG signatures, partial key images for owned outputs, and proofs of legitimacy for those partial images. All of this is sent to the initiator.
        \item {\em Initiator's partial signature}: The initiator verifies the partial transaction's information is valid, and conforms with her expectations (e.g. amounts and recipients are as they should be). She completes the MLSAG signatures and signs them with her private keys, then sends the partially signed transaction to the cosigner along with her revealed opening values.
        \item {\em Finish the transaction}: The cosigner finishes signing the transaction, and submits it to the network.
    \end{enumerate}{}
\end{enumerate}{}

\subsubsection*{Single-commitment signing}

Unlike what is recommended in our multisig chapter, only one commitment is provided per partial transaction (by the transaction initiator), and it is revealed after the cosigner sends out their opening value explicitly. The purpose of committing to opening values (e.g. $\alpha G$) is to prevent a malicious cosigner from using their own opening value to affect the challenge that gets produced, which could allow him to discover other cosigners' aggregation keys (recall Section \ref{sec:threshold-schnorr}). If even one partial opening value is unavailable when the malicious actor generates his own, then it is impossible (or at least negligibly probable) for him to control the challenge.\footnote{This is also why the initiator only reveals her opening values after all transaction information has been determined, so neither signer can alter the MLSAG message and influence the challenge.}\footnote{Single-commitment signing could be generalized as (M-1)-commitment signing, where only the partial transaction author does not commit-and-reveal and other co-signers only reveal after a transaction is fully determined. For example, suppose there is a 3-of-3 address with cosigners (A,B,C), who will attempt single-commitment signing. Signers B and C are in a malicious coalition against A, while C is the initiator and B is the partial transaction author. C initiates with a commitment, then A provides his opening value (without commitment). When B constructs the partial transaction, he can conspire with C to control the signature challenge in order to expose A's private key. Please also note that (M-1)-commitment signing is an original concept first presented here, and is not backed by any advanced research material or code implementation. It may end up being completely erroneous.}\footnote{One way of thinking about this is to consider the meaning and purpose of a `commitment' (recall Section \ref{sec:commitments}). Once Alice commits to value A she is stuck with it, and can't take advantage of new information from event B (caused by Bob) that happens later. Moreover, if A hasn't been revealed then B can't be influenced by it. Alice and Bob can be sure that A and B are independent. It is our contention that single-commitment signing as described meets that standard, and is equivalent to full-commitment signing. If the commitment $c$ is a one-way function of opening values $\alpha_A G$ and $\alpha_B G$ (e.g. $c = \mathcal{H}_n(\alpha_A G,\alpha_B G)$), then if $\alpha_A G$ is committed to initially, $\alpha_B G$ is revealed after $C(\alpha_A G)$ appears, and $\alpha_A G$ is revealed after $\alpha_B G$ appears, then $\alpha_B G$ and $\alpha_A G$ are independent, and $c$ will be random from both Alice's and Bob's perspectives (unless they collaborate, and except with negligible probability).}

Simplifying in this way removes one communication round, which has important consequences for the buyer-vendor interaction experience.


\subsection{Escrowed user experience}
\label{subsec:escrowed-marketplace-escrow-user-experience}

Here is our detailed walk-through of buyer-vendor-moderator interactions involving a 2-of-3 multisig-based online purchase using Monero.
\begin{enumerate}
    \item {\em Buyer makes a purchase}
    \begin{enumerate}
        \item {\em Buyer's new shopping session}: A shopper enters an online marketplace, and her client generates a new subaddress to be used if she starts a new purchase order.\footnote{Using a new subaddress for each purchase order, or even a new subaddress for each separate vendor or vendor's product, makes it more difficult for vendors to track the behavior of their customers. It also helps make sure each purchase order is unique, in the case of someone buying the same thing twice.} In that marketplace she finds vendors, and each vendor is offering a selection of products and prices. Invisible to the shopper herself, but visible to her client (i.e. the software she is using to buy things), each product has a base key for use in multisig-based purchases. Alongside that base key is a list of preselected moderators, and each such moderator has a base key and precomputed vendor-moderator shared secret public key.\footnote{It would be straightforward for vendors to also include, invisible to shoppers, commitments to opening values for transactions. However, to handle multiple purchase orders for the same product, he would have to provide many commitments up front for each potential buyer. We can only imagine how messy that could become. This is part of the utility brought by our single-commitment signing simplification.}
        \item {\em Buyer adds a product to cart}: Our shopper decides she wants something, selects the payment option (i.e. direct payment, 1-of-2 multisig, or 2-of-3 multisig), and if she selects 2-of-3 multisig she is presented with a list of available moderators that she can choose from. When she adds the product to her cart, her client, invisible to her (and assuming she selected 2-of-3 multisig), uses the product's base key, the moderator's base key, and the vendor-moderator shared secret public key, to, in combination with her own shopping-session subaddress's spend key (as her base key), construct a 2-of-3 buyer-vendor-moderator multisig address.\footnote{Exactly how a marketplace should be implemented is open to interpretation, since for example selecting the payment type for a product could be presented to the user at checkout instead of in the `add to cart' interface.}\\

        The view key is a hash of the buyer-vendor shared secret private key (not the aggregation private key, i.e. before {\tt premerge}), and the encryption key for communications between the buyer and vendor is a hash of the view key.\footnote{This same process would take place for 1-of-2 multisig, leaving out the moderator.}
        \item {\em Buyer moves to checkout}: The shopper views her cart with all its products, and decides to proceed to checkout. This is where she makes funding available before finalizing the purchase order. Her client constructs a transaction (but does not sign it yet) that will either pay directly to the vendor, or fund a multisig address (with a little extra for future fees). If funding a 2-of-3 multisig address, the client also initiates two transactions sending money out of that address. One can be used to pay the vendor, and the other to refund the buyer. Their inputs' partial key images are based on the funding transaction that has not been signed yet.\\

        In reality, she only needs committed opening values for the two transactions, and then separately one copy of the partial key images (with proof of legitimacy) and one copy of her shopping-session subaddress. That subaddress has a dual purpose; it is the buyer's address for refunds or change outputs, and its spend key is the buyer's multisig base key.\footnote{It is important to initiate separate transactions, since committed opening values can only be used once.}
        \item {\em Buyer authorizes payment}: After looking over all the purchase order details, the shopper authorizes it.\footnote{If the buyer cancels the purchase order, her funding transaction and partial multisig transactions get deleted.} Her client finishes signing the funding transaction, and submits it to the network.\footnote{If her cart contained multiple vendors' products, then her client can create multiple purchase orders and handle them separately. The vendors can all be paid by the same funding transaction.} It sends the purchase order, along with the funding transaction's hash and initiated multisig transactions, and the buyer-moderator shared secret public key, to the vendor.\footnote{The buyer's client should keep track of payment order details like total price, to later verify the content of multisig transactions before signing them.}
    \end{enumerate}{}
    \item {\em Vendor fulfills a purchase order}
    \begin{enumerate}
        \item {\em Vendor appraises purchase order}: The vendor examines our shopper's purchase order, then approves it for shipping. If he was paid directly then there is nothing else to consider, and if he was paid via 1-of-2 multisig then he can make a transaction paying himself out of that address. For 2-of-3 multisig his client generates a subaddress for receipt of money in the purchase order, and makes two partial transactions out of the buyer's initialized transactions. The payment transaction sends the product price to the vendor, and the rest to the buyer as change, while the refund transaction just empties the multisig address back to the buyer.\footnote{The partial transactions could plausibly share a lot of values since they involve the same inputs, and only one of them should ultimately get signed. For the sake of modularity and robust design we think it's best to handle them separately.} Note that he reconstructs the multisig address out of the buyer-vendor-moderator base keys in combination with the buyer-moderator shared secret public key.
        \item {\em Vendor ships the product}: The vendor ships the product, and sends a fulfillment notification to the buyer. That notification includes a receipt for the purchase, as well as a request for the user to complete payment (everything from here on out refers to 2-of-3 multisig). Hidden to the user is the vendor's purchase order subaddress, which can be used in case of dispute resolution, and both partial transactions.
    \end{enumerate}{}
    \item {\em Buyer completes payment or requests a refund}: A buyer can do this as soon as they receive fulfillment notification, or wait until the product has been delivered.
    \begin{enumerate}
        \item {\em Buyer submits partially signed transaction}: The buyer decides whether to pay for her purchase, or request a refund. Her client creates a partial signature on the appropriate partial transaction, and sends it to the vendor. Any refund would presumably include an explanation justifying it.
        \item {\em Vendor completes transaction}: The vendor receives a partially signed transaction, finishes signing it, and submits it to the network. If necessary he sends a refund notification, with proof, to the buyer.
    \end{enumerate}{}
    \item {\em Moderated dispute}: At any point after the buyer submits a purchase order and before the multisig address is emptied of funds, either the vendor or buyer can decide to get the moderator involved. Party\_A is whoever raised the dispute, while Party\_B is the defendant.\footnote{Our dispute resolution design assumes actors are in good faith. People who are uncooperative, and for example don't initiate or sign transactions that don't favor themselves, will no doubt make the process much more tedious for everyone.}
    \begin{enumerate}
        \item {\em Party\_A contacts moderator}: Party\_A initiates two transactions, for payment or refund, this time geared toward Party\_A-moderator signing, and send them to the moderator along with necessary information for building the multisig address (the base keys, the Party\_A-Party\_B shared secret public key, and the private view key) and reading the multisig wallet's balance (the partial key images and their proofs).
        \item {\em Moderator processes dispute}
        \begin{enumerate}
            \item {\em Moderator acknowledges dispute}: The moderator acknowledges they are looking into the dispute, at the same time creating partial transactions out of the initiated transactions and sending them to Party\_A. They make sure to notify Party\_B about the dispute, and also initiate two more transactions with Party\_B to expedite failure of Party\_A to comply with the final verdict.
            \item {\em Moderator pursues a verdict}: The moderator looks at the available evidence, and can interact with the parties to gather more information. They may try to mediate the argument, in hopes of both parties resolving it without the need to pass a verdict.
            \item {\em Dispute reaches an end}: Either the buyer and vendor resolve it on their own in the end, or the moderator passes a verdict which they communicate to both parties.
        \end{enumerate}{}
        \begin{itemize}
            \item Note: If the defendant party should receive funds per the verdict, but hasn't provided an address for whatever reason, the moderator can try to contact them and cooperate with them to receive those funds. Since it doesn't involve the disputer, that contact can be made (or pursued further) after the dispute has been resolved.
        \end{itemize}{}
        \item {\em Party\_A or B accepts the verdict}: If no Party\_A-Party\_B transactions were finalized, it implies the dispute was resolved by moderator's decision.
        \begin{enumerate}
            \item {\em Party\_A accepts}
            \begin{enumerate}
                \item Party\_A completes their partial signature on the verdict transaction, and sends it to the moderator.
                \item The moderator completes the signature, and submits the transaction.
            \end{enumerate}{}
            \item {\em Party\_B accepts}
            \begin{enumerate}
                \item Party\_B creates a partial transaction for the moderator's initialized verdict transaction, and sends it to the moderator. This step can be performed before the verdict is finalized, in which case Party\_B would make partial transactions for both potential verdicts.
                \item The moderator partially signs that partial transaction, and sends it to Party\_B.
                \item Party\_B finishes signing the transaction, and submits it to the network. He sends the transaction hash to the moderator.
            \end{enumerate}{}
        \end{enumerate}{}
        \item {\em Moderator closes out the dispute}: The moderator summarizes the dispute and its resolution, and sends their report to the buyer and vendor.
    \end{enumerate}{}
\end{enumerate}{}

There are four key design optimizations installed.

\subsubsection*{Preselected moderators}

By choosing the moderators ahead of time, vendors can create a shared secret with each of them for each of their products and publish its public key with the product information.\footnote{Importantly, these multisig addresses are still resistant to key aggregation tests since the buyer's shared secrets are unknown to observers.} This way buyers can construct the complete merged multisig address in one step, as soon as they decide to purchase something, harmonizing with the requirement for `offline selling'. Preselecting multiple moderators allows buyers to choose the one they trust more.

Buyers may also, if they don't trust a vendor's accepted moderators, cooperate with an online moderator of their choosing to make a multisig address with the vendor's product base key. After receiving a purchase order, the vendor may accept that new moderator or decline the sale.\footnote{For expediency, an escrow service could be `always online', and instead of using preselected moderators all 2-of-3 multisig addresses are actively constructed with that service when a purchase order is made. Another possibility is using nested multisig (Section \ref{sec:general-key-families}), where the preselected moderator is actually a 1-of-N multisig group. That way whenever a dispute arises any moderator from that group who happens to be available can step in. It would likely require substantial development effort to implement.}

We expect the mutual desire of buyers and vendors for good moderators to, over time, create a hierarchy of moderators organized by the quality and fairness of the service they provide. Lower quality moderators or those with less of a reputation would likely earn less money or service fewer or less significant transactions.\footnote{It is not clear to us the best, or most likely, funding method for moderators. Perhaps they would get paid a flat or percentage fee of each moderated transaction or transaction where they are added as a moderator (and then if the fee wasn't provided in the original partial transactions, refuse to cooperate in a dispute), or users and/or vendors and/or marketplace platforms would contract with them.}

\subsubsection*{Subaddresses and product IDs}

Vendors create a new base key for each product line/ID, and those keys are used to construct 2-of-3 multisig addresses.\footnote{This base key is also used for 1-of-2 multisig purchasing. We feel it is important not to expose the private spend key in the communication channel, so using a buyer-vendor shared secret makes a lot of sense.} When vendors fulfill a purchase order they create a unique subaddress for receipt of funds, which can be used to match purchase orders with payments received.

The requirement for `purchase order-based payments' is met efficiently here, especially since funds directed to different subaddresses are trivially accessible from the same wallet (recall Section \ref{sec:subaddresses}).

\subsubsection*{Anticipatory partial transactions}

Multisig transactions take multiple rounds, so we begin making them as soon as possible. For user convenience, partial transactions that are only rarely used (e.g. refund transactions) get made early, so they are immediately available for signing if the need arises.

\subsubsection*{Conditional moderator access}

For the sake of both efficiency and privacy, moderators only need access to the details of a trade when settling disputes. To accomplish this, we make the multisig private view key a hash of the buyer-vendor shared secret private key $k^{v,grp}_{purchase-order} = \mathcal{H}_n(T_{mv},k^{sh,\textrm{2x3}}_{AB})$ where $T_{mv}$ is the marketplace view key domain separator and A and B correspond with vendor and buyer, and $k^{sh,\textrm{2x3}}_{AB} = \mathcal{H}_n(k^{base}_{A}*k^{base}_{B} G)$. We extend what it can `view' to the buyer-vendor communication transcript. In other words, the communication encryption key is $k^{ce}_{purchase-order} = \mathcal{H}_n(T_{me},k^{v,grp}_{purchase-order})$ ($T_{me}$ is the marketplace encryption key domain separator).\footnote{Separating the view key and encryption key allows handing out just view rights to the communication log without view rights to the multisig address's transaction history.}\footnote{This method is also used for 1-of-2 multisig addresses.}

Moderators gain access to the buyer-vendor communications, and the ability to authorize payments, only when one of the original parties releases the view key to them.\footnote{It is important for moderators to verify the communication log they receive hasn't been tampered with. One way might be for each cosigner to include a signed hash of the current message log whenever they send out a new message. Moderators can look at the back-and-forth, and series of hashed logs, to identify discrepancies. It would also help cosigners identify messages that failed to transmit, and alternatively create evidence that particular signers did in fact receive certain messages.}

Moreover, vendors can verify the marketplace host (who may also be the only moderator available, depending on how this concept is implemented) is not MITM (`man in the middle') of their conversations with customers (i.e. pretending to be the buyer or vendor) by checking that the base keys they publish for each product match what gets displayed. Since the buyer's base key, which gets used to create the multisig address, is also part of the encryption key, a malicious host would have a hard time orchestrating a MITM attack.