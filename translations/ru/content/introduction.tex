\chapter{Введение}
\label{chapter:introduction}

В цифровой сфере сплошь и рядом создаются бесконечные копии информации с таким же бесконечным количеством изменений. Чтобы валюта смогла существовать в цифровом виде и получила широкое распространение, её пользователи должны верить в то, что её денежная масса строго ограничена. Получатель денег должен быть уверен, что он не получит поддель\-ных монет или монет, которые уже были отправлены кому-то другому. Чтобы добиться этого без привлечения какой-либо третьей стороной, например, центрального руководящего органа, информация о денежной массе и полная история транзакций должны быть доступны для публичной проверки.

Мы можем использовать криптографические инструменты, чтобы регистрировать такую ин\-формацию в доступной базе данных, блокчейне, где она будет оставаться практически неиз\-менной, где её будет невозможно подделать, а уровень её легитимности не сможет оспорить никакая из сторон.
\\ \newline
Данные криптовалютных транзакций сохраняются в блокчейне, который служит публичным реестром\footnote{В этом контексте реестр означает запись всех событий, связанных с созданием и обменом валюты. В частности, сколько денег было переведено в рамках каждого события и кому.} всех валютных операций. Большинство криптовалют сохраняет данные своих транзакций в виде простого текста, что упрощает процесс проверки (верификации) таких транзакций сообществом.
\\ \newline
Очевидно, что открытый блокчейн не соответствует каким-либо принципам анонимности или взаимозаменяемости\footnote{«\textbf{Взаимозаменяемость} означает возможность взаимного замещения при использовании или выполнении контракта. ... Примеры: ... деньги и т. д.»\cite{mises-org-fungible} В случае с открытым блокчейном, таким как блокчейн Bitcoin, монеты, которыми владеет Элис, могут отличаться от монет, которыми владеет Боб из-за `истории транзакций' таких монет. Если история транзакций монет Элис содержит информацию о транзакциях, предположительно связанных с незаконными действиями, её монеты могут быть `помечены' \cite{bitcoin-big-bang-taint}, а следовательно, они будут менее ценными, чем монеты Боба (даже если это будет одна и та же сумма монет). Согласно заслуживающим доверия цифрам вновь созданные монеты Bitcoin торгуются с наценкой, если сравнивать их с монетами, имеющими историю, так как за ними просто не стоит никакой истории \cite{new-bitcoin-premium}.}, так как он буквально {\em придаёт огласке} полную историю всех транзак\-ций, совершаемых его пользователями.
\\ \newline
Для решения проблемы недостаточной приватности пользователи таких криптовалют, как Bitcoin, могут «маскировать» свои транзакции, используя временные промежуточные адреса \cite{DBLP:journals/corr/NarayananM17}. Тем не менее, обладая соответствующими инструментами, можно проанализировать потоки и с большой степенью вероятности связать истинных отправителей с получателями \cite{DBLP:journals/corr/ShenTuY15b, DK-police-tracing-btc, Andrew-Cox-Sandia, chainalysis-2020-report}.

В отличие от таких валют Monero (Moe-neh-row) пытается решить проблему приватности путём сохранения в блокчейне только скрытых одноразовых адресов получателей средств. При этом подтверждение распределения средств в каждой транзакции осуществляется при помощи кольцевых подписей. Благодаря применению этих методов пока не было найдено никаких известных способов связать отправителей с получателями или отследить источник средств.\footnote{В зависимости от поведения пользователей могут возникнуть ситуации, в которых транзакции можно будет в некоторой степени проанализировать. Примеры можно найти в этой статье: \cite{monero-ring-heuristics-ryo}.}

Помимо этого, суммы транзакций в блокчейне Monero скрыты криптографическими конструк\-циями, которые обеспечивают непрозрачность валютных потоков.

Результатом является высокий уровень приватности и взаимозаменяемости криптовалюты.



\section{Цели}
\label{sec:goals}

Monero — это устоявшаяся криптовалюта, история разработки которой насчитывает уже более пяти лет \cite{bitmonero-launched, monero-history}, и уровень распространённости которой постоянно растёт \cite{justin-defcon-2019-community-growth}.\footnote{\label{marketcap_note}С точки зрения рыночной капитализации, показатели Monero были стабильными относительно других криптовалют. По состоянию с 14 июня 2018 г. по 5 января 2020 г. Monero занимала 14-ю, позицию; см. \url{https://coinmarketcap.com/}.} К сожалению, документации, подробно описывающей механизмы, используемые этой валютой, практически нет.\footnote{В одном из документов, опубликованных на \url{https://monerodocs.org/}, можно найти полезную информацию, в частности, связанную с интерфейсом командной строки (CLI). CLI-кошелёк поддерживает ввод команд посредством командной строки / консоли. Этот кошелёк имеет больше всего функций в сравнении с другими кошельками Monero, но они реализованы за счёт отсутствия удобного графического пользовательского интерфейса.}\footnote{Другой документ более общего характера под названием «Осваиваем Monero (Mastering Monero)» можно найти здесь: \cite{mastering-monero}.} Хуже того, важные части теоретической концепции были опубликованы в работах, не прошедших независимой технической экспертизы. Такие работы нельзя считать полными. Помимо этого, в них содержатся ошибки. В большинстве случаев надёжным источником информации с точки зрения теоретической концепции Monero может служить только исходный код.\footnote{Г-ном Сегиасом (Seguias) была написана превосходная серия статей под названием «Структурные элементы Monero» (Monero Building Blocks) \cite{monero-building-blocks}, в которых подробно рассмотрены криптографические доказательства безопасности, используемые для обоснования схем подписей Monero. Как и в случае с первой редакцией «От нуля к Monero: Первое издаине» (Zero to Monero: First Edition) \cite{ztm-1}, серия статей Сегиаса сфокусирована на седьмой версии протокола.}

Кроме того, для тех, кто не имеет математического образования, изучение основ криптогра\-фии на эллиптических кривых, которую широко использует Monero, может оказаться совер\-шенно бесцельным и разочаровывающим предприятием.\footnote{В рамках предыдущей попытки объяснить, как работает Monero \cite{MRL-0003-about-monero}, не была охвачена криптография эллиптических кривых, то есть работа была неполной, и теперь, спустя пять лет, её можно считать устаревшей.}

Мы намерены исправить сложившуюся ситуацию, изложив фундаментальные концепции, необходимые для понимания криптографии на эллиптических кривых, рассмотрим алгорит\-мы и криптографические схемы, а также соберём воедино подробную информацию о внутрен\-них механизмах работы Monero.

Чтобы нашим читателям было удобнее, мы постарались составить конструктивное, пошаговое описание криптовалюты Monero.

Во второй редакции этого отчета мы сосредоточили своё внимание на двенадцатой версии протокола Monero\footnote{`Протокол' представляет собой набор правил, согласно которым каждый блок тестируется перед тем, как он будет добавлен в блокчейн. Этот набор правил включает в себя `протокол транзакций' (в настоящее время его вторую версию, RingCT) — это общие правила, определяющие принципы построения транзакций. Определённые правила, связанные с транзакциями, могут изменяться и изменяются без изменения версии протокола транзакций. Только масштабные изменения в структуре транзакций подразумевают изменение номера версии.}, что соответствует версии 0.15.x.x программного пакета Monero. Все описанные здесь механизмы, связанные с транзакциями и блокчейном, относятся именно к этим версиям.\footnote{Целостность и надёжность кодовой базы Monero обеспечиваются тем, что они были проанализированы достаточным количеством человек, выявивших все значительные ошибки. Мы надеемся, что читатели не воспримут наши объяснения как нечто, не требующее доказательств, и самостоятельно убедятся в том, что код работает именно так, как предполагается. Если будет иначе, мы надеемся, что вы ответственно раскроете эту информацию (\url{https://hackerone.com/monero}) , если это будет касаться каких-то серьёзных проблем, или создадите пул-реквест на GitHub (\url{https://github.com/monero-project/monero}), если речь будет идти о минимальных недочётах.}\footnote{В настоящий момент исследуется и анализируется несколько протоколов, заслуживающих рассмотрения в связи с транзакциями Monero следующего поколения. Это такие протоколы, как Triptych \cite{triptych-preprint}, RingCT3.0 \cite{ringct3-preprint}, Omniring \cite{omniring-paper}, и Lelantus \cite{lelantus-preprint}.} Устаревшие схемы транзакций не анализировались нами, даже несмотря на то, что они могут частично поддерживаться из соображений обратной совместимости. То же относится и к устаревшим функциям блокчейна. Первая редакция \cite{ztm-1} относится к седьмой версии протокола и версии 0.12.x.x программного пакета.



\section{Читателю}

Мы ожидаем, что многие читатели, столкнувшиеся с этим отчетом, практически не будут иметь представления о том, что такое дискретная математика, алгебраические структуры, криптография\footnote{Исчерпывающий учебник по прикладной криптографии можно найти здесь: \cite{applied-cryptography-textbook}.} и блокчейны. Мы постарались достаточно подробно представить материал, чтобы непрофессионалы с любой точки зрения могли изучить Monero без проведения каких-либо внешних исследований.

Нами были намеренно пропущены или перенесены в сноски некоторые математические и технические детали, если они были необходимы для ясности. Нами также были пропущены конкретные подробности реализации в тех местах, где мы не сочли их важными. Наша цель состояла в том, чтобы представить материал на стыке математической криптографии и компьютерного программирования. При этом должны были сохраниться полнота информа\-ции и чёткость изложения концепции.\footnote{Сноски, содержащиеся в некоторых главах, особенно в главах, связанных с протоколом, пересекаются с последующими главами и разделами. Они призваны сделать их понятнее при повторном прочтении, так как содержат некоторые подробности реализации, которые будут полезны тем, кто желает разобраться в том, как работает Monero.}



\section{Истоки криптовалюты Monero}

Криптовалюта Monero, которая изначально называлась BitMonero, была создана в апреле 2014 в качестве производной криптовалюты на базе протокола доказательства концепции CryptoNote \cite{bitmonero-launched}. Monero на языке эсперанто означает «деньги», а форма множественного числа звучит как Moneroj (Moe-neh-rowje, похоже на Moneros, но с -ge на конце, как в английском слове orange).

CryptoNote является криптовалютным протоколом, разработанным самыми разными людьми. Первой значимой работой, в которой был описан этот протокол, стал документ, опубликован\-ный в октябре 2013 \cite{cryptoNoteWhitePaper}. Автор документа, который скрывался под псевдонимом Николас Ван Саберхаген (Nicolas van Saberhagen), предложил обеспечивать анонимность отправителей при помощи одноразовых адресов, а неотслеживаемость отправителей путём применения кольцевых подписей.

После этого аспекты приватности Monero усиливались за счёт реализации возможности сокры\-тия сумм, что описано Грэгом Максвеллом (Greg Maxwell) и другими авторами в работе \cite{Signatures2015BorromeanRS}, а также улучшения кольцевых подписей в соответствии с рекомендациями Шена Ноезера (Shen Noether) \cite{MRL-0005-ringct}, и в результате эффективность подписей повысилась благодаря реализации Bulletproofs. \cite{Bulletproofs_paper}.



\section{Структура документа}

Как уже было сказано ранее, нашей целью является создание полного и последовательного описания криптовалюты Monero. Структура «От нуля к Monero» соответствует этой задаче. Читатель последовательно знакомится со всеми подробностями внутренних принципов работы валюты.


\subsection{Часть 1: «Основы»}

Для обеспечения полноты описания мы решили представить все базовые криптографические элементы, необходимые для понимания сложностей, связанных с Monero, а также их \linebreak математические основы. Во \ref{chapter:basicConcepts} главе нами рассматриваются важные аспекты криптографии на эллиптических кривых.

В \ref{chapter:advanced-schnorr} главе более широко рассмотрена схема подписи Шнорра, которой мы коснулись в предыду\-щей главе, а также кратко описаны алгоритмы построения кольцевых подписей, используемые для обеспечения конфиденциальности транзакций.

В \ref{chapter:addresses} главе описано, как Monero использует адреса для контроля над обладанием средствами, а также рассматриваются различные виды адресов.

В \ref{chapter:pedersen-commitments} главе нами приводятся криптографические механизмы сокрытия сумм.

Наконец, после описания всех компонентов можно перейти и к схемам самих транзакций, используемых Monero, и \ref{chapter:transactions} глава посвящена именно им.

Подробности, связанные с блокчейном Monero, раскрываются в \ref{chapter:blockchain} главе.


\subsection{Часть 2: «Дополнения»}

Криптовалюта — это больше, чем просто протокол, и в части «Дополнения» мы нами рассма\-тривается ряд различных идей, многие из которых не были реализованы.\footnote{Следует отметить, что будущие версии протокола Monero, в частности, реализующие новые протоколы транзакций, могут сделать воплощение этих идей невозможным или непрактичным.}

Различная информация, связанная с транзакциями, может быть доказана перед наблюдателя\-ми, и соответствующие методы описаны в \ref{chapter:tx-knowledge-proofs}главе.

Хотя это и не важно с точки зрения работы Monero, мультиподписи очень полезны и позволя\-ют множеству людей совместно отправлять и получать деньги. В \ref{chapter:multisignatures} главе описан текущий подход Monero к применению мультиподписей, и кратко описаны возможные будущие разра\-ботки в этой области.%This is formally called (N-1)-of-N and N-of-N threshold authentication.

Чрезвычайно важно применять схему multisig при взаимодействии продавцов и покупателей на торговых онлайн площадках. В \ref{chapter:escrowed-market} главе показан пример нашего первоначального решения торговой эскроу-площадки с использованием схемы multisig, применяемой Monero.

В этом отчёте в \ref{chapter:txtangle} главе мы впервые представляем TxTangle, децентрализованный протокол объединения транзакций множества пользователей в одну.


\subsection{Приложения}

В Приложении \ref{appendix:RCTTypeBulletproof2} рассматриваются примеры структур транзакций, взятых из блокчейна. В Приложении \ref{appendix:block-content} разъясняется структура блоков (включая их заголовки и транзакции майнеров) в блокчейне Monero. Наконец, Приложение \ref{appendix:genesis-block} завершает наш отчёт объяснением структуры генезис-блока Monero. Так мы попытались связать теорию, изложенную в предыдущих частях работы, с её реализацией на практике.

Мы\marginnote{Взгляните, разве это не полезно?} используем примечания на полях, чтобы указать, где в исходном коде можно найти детали реализации Monero.\footnote{Наши сноски на полях точны с точки зрения версии 0.15.x.x программного пакета Monero, но постепенно могут утратить свою точность, так как кодовая база постоянно меняется. Тем не менее код сохраняется в соответствующем репозитории GitHub (\url{https://github.com/monero-project/monero}), что обеспечивает постоянный доступ к истории изменений.} Как правило, существует путь к файлу, например src/ringct/rctOps.cpp, и функция, например \(\textrm{{\tt ecdhEncode()}}\). Примечание: `-' обозначает разделенный текст, например, crypto- note $\rightarrow$ cryptonote, и в большинстве случаев мы игнорируем операторы пространства имен (например {\tt Blockchain::}).



\section{Отказ от ответственности}

Все схемы подписей, варианты применения эллиптических кривых и подробности реализации Monero носят исключительно описательный характер. Читателям, исследующим серьёзные варианты практического применения (в отличие от исследователей-любителей), следует ис\-пользовать первоисточники и технические спецификации (которые мы цитировали, где было возможно). Схемы подписей нуждаются в хорошо проверенных доказательствах безопасности, а детали реализации Monero можно найти в исходном коде Monero. В частности, как гласит поговорка, «не выдумывайте своей собственной криптографии». Криптографические прими\-тивы, реализующие код, должны быть тщательно проанализированы экспертами и иметь долгую историю надёжной работы. Кроме того, оригинальные статьи, на которые ссылается этот документ, могут быть проанализированы в недостаточной мере и, вероятно, не проверены, поэтому читателям следует проявлять своё собственное суждение при их прочтении.



\section{История «От нуля к Monero»}

«От нуля к Monero» является как бы расширением магистерской диссертации Курта Алонсо под названием «Monero – технология обеспечения приватности блокчейна» \cite{kurt-original}, которая была опубликована в мае 2018 года. Первая редакция была опубликована в июне 2018 года \cite{ztm-1}.

Во второй редакции нами была улучшена часть, посвящённая применению кольцевых подпи\-сей (Глава \ref{chapter:advanced-schnorr}), реорганизовано описание транзакций (добавлена Глава \ref{chapter:addresses}, посвящённая адресам Monero), обновлено описание метода передачи сумм выходов (раздел \ref{sec:pedersen_monero}), ольцевые подписи Борромео заменены на Bulletproofs (раздел \ref{sec:range_proofs}), исключены {\tt RCTTypeFull} (Глава \ref{chapter:transactions}), обновлено и доработано описание динамических весовых значений блоков Monero и системы комиссий (Глава \ref{chapter:blockchain}), рассмотрены доказательства, связанные с транзакциями (Глава \ref{chapter:tx-knowledge-proofs}), описаны мульти\-подписи Monero (Глава \ref{chapter:multisignatures}), предложены решения по реализации торговых эскроу-площадок (Глава \ref{chapter:escrowed-market}), предложен новый протокол децентрализованных объединённых транзакций под названием TxTangle (Глава \ref{chapter:txtangle}), с целью обеспечения соответствия текущей версии протокола (v12) и программному пакету Monero (v0.15.x.x) обновлены или добавлены различные незна\-чительные подробности. Также из соображений удобочитаемости «вычищен» весь документ в целом.\footnote{Исходный код \LaTeX{} для обеих редакций «От нуля к Monero» можно найти здесь (первая редакция находится в ветке `ztm1'): \url{https://github.com/UkoeHB/Monero-RCT-report}.}



\section{Благодарность}
\label{sec:acknowledgements}

Написано автором `koe'.

Создание данного отчёта было бы невозможным без оригинальной магистерской диссертации Курта \cite{kurt-original}, и именно ему первому я выражаю благодарность. Исследователи из Исследова\-тельской лаборатории Monero (MRL), Брэндон Гуддэл (Surae Noether), и псевдо-анонимный Саранг Ноезер (Sarang Noether) с которым мы сотрудничали при работе над разделом \ref{sec:range_proofs} и Главой \ref{chapter:tx-knowledge-proofs}), стали надёжным и исчерпывающим источником знаний при написании обеих редакций «От нуля к Monero». `moneromooo', , самый плодовитый из ведущих разработчиков Проекта Monero, обладающий, пожалуй, самым глубоким знанием кодовой базы на этой планете, множество раз подталкивал меня в нужном направлении. И, конечно же, множество других замечательных контрибьюторов Monero потратило немало времени, отвечая на мои бесконеч\-ные вопросы. Наконец, хотелось бы поблагодарить тех многих людей, которые связывались с нами, чтобы внести коррективы — спасибо вам за ваши полезные и ободряющие комментарии!