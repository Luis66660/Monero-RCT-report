\chapter{Monero на рынке эскроу-услуг}
\label{chapter:escrowed-market}

В большинстве случаев покупки в интернет-магазинах проходят гладко. Покупатель отправ\-ляет деньги продавцу, а затем ожидаемый товар доставляется прямо к его порогу. Если у товара имеется какой-либо дефект или же, что происходит в некоторых случаях, покупатель меняет своё решение, товар может быть возвращён, а покупатель получит соответствующее возмещение.

Трудно доверять человеку или организации, с которыми вы не были знакомы ранее, и многие покупатели чувствуют себя в безопасности при совершении покупок только потому, что знают, что компания, выпустившая их кредитную карту, сможет отменить проведённый платеж по запросу \cite{credit-card-reversals}.

Криптовалютные транзакции необратимы, и существует лишь ограниченное количество слу\-чаев, в которых покупатель или продавец могут воспользоваться средствами юридической защиты, если что-то пойдёт не так. Особенно это касается Monero, анализ блокчейна которой связан с определёнными трудностями \cite{chainalysis-2020-report}. Краеугольным камнем с точки зрения надёжности любого интернет-магазина, принимающего в качестве оплаты криптовалюты, является воз\-можность прибегнуть к услугам эскроу-сервиса, использующего мультиподписи, которые строятся по схеме «2 из 3». Это позволяет третьим сторонам выступать в качестве посредника при разрешении споров. Доверяя этим третьим лицам, даже совершенно анонимные продавцы и их покупатели могут взаимодействовать без каких-либо формальностей.

Поскольку взаимодействие с использованием мультиподписей Monero подразумевает наличие определённых сложностей (см. подпункт \ref{sec:simplified-communication}), в этой главе будет говориться о максимально эффективной среде для торговли посредством эскроу-сервисов.\footnote{Рене Бруннер ``rbrunner7" является контрибьютором Monero, создавшим MMS \cite{mms-project-proposal, mms-manual}. Именно им были проведены исследования в отношении возможности интеграции схем мультиподписи Monero в децентрализованную криптовалютную торговую площадку OpenBazaar \url{https://openbazaar.org/}. Изложенные здесь концепции непосредственно связаны с теми препятствиями, с которыми столкнулся Рене \cite{openbazaar-rbrunner-investigation} (на «ранних этапах анализа»).}\footnote{Как нам кажется, в текущей реализации схемы мультиподписи Monero процесс создания транзакций уже аналогичен тому, который необходим для торговли с привлечением эскроу-сервисов, что представляется хорошей новостью с точки зрения потенциальной работы по её внедрению. Читателю следует обратить внимание на то, что схема мультиподписи Monero требует некоторого обновления в плане безопасности, прежде чем её можно будет широко использовать \cite{multisig-research-issue-67}.}



\section{Важные особенности}
\label{sec:escrowed-marketplace-essential-features}

Существует несколько основных требований и функций, упрощающих онлайн взаимодействие между покупателем и продавцом. Нами были использованы данные исследования OpenBazaar, проведённого Рене \cite{openbazaar-rbrunner-investigation}, поскольку они вполне разумны и расширяемы.
\begin{itemize}
    \item {\em Офлайн продажи}: покупателю требуется доступ к онлайн-магазину продавца, чтобы разместить заказ, даже если продавец находится вне сети. Очевидно, что продавец должен войти в сеть, чтобы получить заказ и выполнить его.\footnote{Выполнение заказа подразумевает отправку товара, который будет доставлен покупателю.}
    \item {\em Платежи на основе заказа}: адреса продавцов для получения средств уникальны для каждого отдельного заказа. Это позволяет обеспечить соответствие заказов вносимым платежам.
    \item {\em Покупка при высоком уровне доверия}: покупатель, если он по-настоящему доверяет продавцу, может заплатить за товар ещё до того, как заказ будет выполнен или достав\-лен.
    \begin{itemize}
        \item {\em Прямые онлайн-платежи}: убедившись в том, что продавец находится в сети и список его товаров доступен, покупатель отправляет продавцу деньги в одной транзакции на адрес, указанный продавцом, который затем сообщает покупателю, что заказ выполняется.
        \item {\em Офлайн платежи}: если продавец находится вне сети, покупатель создаёт multisig-адрес по схеме «1 из 2» и перечисляет на него средства, достаточные, чтобы по\-крыть стоимость предполагаемой покупки. Когда продавец войдёт в сеть, он смо\-жет снять деньги с multisig-адреса (отправив сдачу обратно покупателю) и выпол\-нить заказ. Если продавец более уже не вернётся в сеть (или, например, по истече\-нии некоторого разумного периода ожидания) или же если покупатель передумает, прежде чем вернётся продавец, покупатель сможет перевести средства обратно с адреса, созданного по схеме «1 из 2» обратно на свой личный кошелек.
    \end{itemize}{}
    \item {\em Модерируемая покупка}: покупателем, продавцом и модератором создаётся multisig-адрес по схеме «2 из 3». Модератор при этом согласуется как покупателем, так и продавцом. Покупатель отправляет средства на этот адрес до выполнения своей части сделки про\-давцом, а затем, после доставки товара, обе стороны участвуют в процессе «разблоки\-ровки» средств. Если продавец и покупатель не смогут прийти к соглашению, они могут обратиться к модератору для разрешения спорной ситуации.
\end{itemize}

Мы не рассматриваем покупки при высоком уровне доверия, поскольку ввиду отсутствия необходимости в сложном взаимодействии процесс будет довольно тривиальным.\footnote{Мультиподпись, построенная по схеме «1 из 2», может использовать преимущества некоторых концепций, полезных для создания мультиподписи по схеме «2 из 3», в частности, построение адреса в первую очередь.}


\subsection{Процесс покупки}
\label{subsec:escrowed-marketplace-purchasing-workflow}

Все покупки должны совершаться в соответствии с одним и тем же набором шагов при условии, что все стороны проявят при этом должную осмотрительность. Количество шагов зависит от того, с чем столкнётся модератор при вступлении в игру, например: «Вы запраши\-вали возврат средств, прежде чем привлечь меня?».
\begin{enumerate}
    \item Покупатель получает доступ к онлайн-магазину продавца, выбирает товар, который собирается приобрести, выбирает опцию «Я хочу это приобрести», обеспечивает доступ к средствам, необходимым для покупки этого товара, а затем отправляет заказ продавцу.
    \item Продавец получает заказ, проверяет, есть ли товар на складе и достаточно ли средств имеется у покупателя, и либо возвращает покупателю деньги, либо выполняет заказ, отправляя товар и уведомив об этом покупателя соответствующей квитанцией.
    \begin{itemize}
        \item В случае с мультиподписью, созданной по схеме «2 из 3», покупатель может опцио\-нально подтвердить платёж при получении уведомления о выполнении заказа.
    \end{itemize}{}
    \item Покупатель либо получает товар должным образом, либо не получает товар вовремя, либо же получает товар с дефектом.
    \begin{itemize}
        \item {\em Хороший товар}: покупатель либо оставляет отзыв для продавца, либо вообще не делает ничего.
        \begin{itemize}
            \item {\em Покупатель оставляет отзыв}: покупатель может оставить как положитель\-ный, так и отрицательный отзыв.
            \begin{itemize}
                \item {\em Положительный отзыв}: если используется схема мультиподписи «2 из 3» и товар ещё не был оплачен, то на этом этапе покупатель подтверждает перевод оплаты продавцу. В ином случае это просто положительный отзыв. [КОНЕЦ ПРОЦЕССА]
                \item {\em Отрицательный отзыв}: если используется схема мультиподписи «2 из 3», то на этом этапе происходит переход к процессу «плохой товар». В ином случае это просто отрицательный отзыв.
            \end{itemize}{}
            \item {\em Покупатель ничего не делает}: либо продавцу уже заплатили, либо использует\-ся схема мультиподписи «2 из 3» и для получения средств ему требуется опре\-делённое взаимодействие с кем-то.
            \begin{itemize}
                \item {\em Продавцу заплатили}: [КОНЕЦ ПРОЦЕССА]
                \item {\em Продавцу не заплатили}: продавец требует оплаты.
                \begin{enumerate}
                    \item Продавец связывается с покупателем и запрашивает платёж (или отправ\-ляет напоминание).
                    \item Покупатель отвечает, либо не отвечает.
                    \begin{itemize}
                        \item {\em Покупатель отвечает}: ответ покупателя может быть либо платежом, либо запросом возмещения.\\
                        $>$ {\em Покупатель производит платёж}. [КОНЕЦ ПРОЦЕССА]\\
                        $>$ {\em Покупатель запрашивает возврат средств}: переход к процессу\linebreak «плохой товар».
                        \item {\em Покупатель не отвечает}: продавец обращается к модератору с целью «разблокировки» средств. [КОНЕЦ ПРОЦЕССА]
                    \end{itemize}{}
                \end{enumerate}{}
            \end{itemize}
        \end{itemize}{}
        \item {\em Отсутствие товара или плохой товар}: покупатель запрашивает возврат средств.
        \begin{enumerate}
            \item Покупатель связывается с продавцом и делает запрос о возврате средств, воз\-можно, с объяснением причины.
            \item Продавец выполняет запрос на возврат средств или оспаривает его (или же просто игнорирует его).
            \begin{itemize}
                \item {\em Продавец выполняет запрос}: деньги возвращаются к покупателю. [КОНЕЦ ПРОЦЕССА]
                \item {\em Продавец отказывается удовлетворить запрос}: используется либо схема multisig «2 из 3», либо иная схема.
                \begin{itemize}
                    \item {\em Используется иная схема от multisig «2 из 3»}: [КОНЕЦ ПРОЦЕССА]
                    \item {\em Используется схема multisig «2 из 3»}: либо покупатель отказывается от запроса на возврат средств (буквально или же он просто не может ответить своевременно), либо покупатель настаивает на возврате.
                    \begin{itemize}
                        \item {\em Покупатель отказывается от запроса}: либо покупатель подтверждает платёж, либо нет.\\
                        $>$ {\em Покупатель подтверждает платёж}: [КОНЕЦ ПРОЦЕССА]\\
                        $>$ {\em Покупатель не подтверждает платёж}: продавец связывается с модератором, который подтверждает платёж. [КОНЕЦ ПРОЦЕССА]
                        \item {\em Покупатель настаивает на возмещении}: продавец или покупатель связывается с модератором, который обсуждает возникшую проблему вместе с участниками для вынесения решения. [КОНЕЦ ПРОЦЕССА]
                    \end{itemize}{}
                \end{itemize}
            \end{itemize}{}
        \end{enumerate}{}
    \end{itemize}{}
\end{enumerate}{}



\section{Бесшовные мультиподписи Monero}
\label{sec:escrowed-marketplace-seamless-multisig}

Мы можем воспользоваться преимуществом, которое имеется при обычном заказе на приобре\-тение товара, и «сжато» пройти почти все этапы взаимодействия с использованием мульти\-подписи Monero, построенной по схеме «2 из 3», так, что участники этого даже не заметят. Есть ещё один маленький шаг, который должен будет сделать продавец в конце. Ему будет необходимо подписать и представить окончательную транзакцию для получения оплаты (по аналогии с «изъятием наличных из кассового аппарата»).\footnote{Расширение этой схемы за пределы «(N-1) из N», вероятно, невозможно без введения дополнительных шагов из-за наличия дополнительных раундов, необходимых для создания multisig-адреса с более низким пороговым значением.}


\subsection{Основные принципы взаимодействия в рамках схемы multisig}
\label{subsec:escrowed-marketplace-multisig-interaction-basics}

Все взаимодействия в рамках схемы multisig «2 из 3» предполагают наличие одинакового набора раундов обмена данными, включая создание адреса и построение транзакции. Для обеспечения соответствия нормальному рабочему процессу мы реорганизуем процесс построе\-ния транзакции, если сравнивать с тем, как это описано в главе, посвящённой мультиподписям (см. подпункты \ref{sec:simplified-communication} и \ref{sec:n-1-of-n}).\footnote{Данная процедура на самом деле очень похожа на то, как в настоящее время в Monero организуются транзакции с мультиподписями.}
\begin{enumerate}
    \item {\em Создание адреса}
    \begin{enumerate}
        \item Прежде всего, все пользователи должны узнать базовые ключи других участников, которые будут использоваться ими для построения общих секретов. Затем они должны передать публичные ключи этих общих секретов (например,\linebreak $K^{sh} = \mathcal{H}_n(k^{base}_A*k^{base}_B G) G$) другим пользователям.
        \item После того как все общие секретные публичные ключи будут известны, каждый пользователь может произвести предварительное слияние (функция {\tt premerge}а за\-тем окончательно объединить их (функция {\tt merge}) в публичный ключ траты адреса. Агрегированные приватные ключи траты будут использоваться для подписания транзакций. Хеш общего секретного приватного ключа основных подписантов, покупателя и продавца, (например, $k^{sh} = \mathcal{H}_n(k^{base}_A*k^{base}_B G)$) будет использоваться в качестве ключа просмотра (например, $k^v = \mathcal{H}_n(k^{sh})$) будет использоваться в качестве ключа просмотра (например, \ref{subsec:escrowed-marketplace-escrow-user-experience}).
    \end{enumerate}{}
    \item {\em Построение транзакции}: предположим, что по адресу уже имеется хотя бы один выход, а образы ключей неизвестны. Есть два подписанта: инициатор и соподписант.
    \begin{enumerate}
        \item {\em Запуск создания транзакции}: инициатор решает создать транзакцию. Он генери\-рует начальные значения (например, $\alpha G$) для всех имеющихся у него выходов (он ещё не уверен, какие из них будут использованы) и создаёт по ним обязательства. Он также создаёт частичные образы ключей для принадлежащих ему выходов и подписывает их при помощи доказательства легитимности (см. подпункт \ref{sec:recalculating-key-images-multisig}). Затем он отправляет эту информацию вместе со своим личным адресом для получе\-ния средств (например, для получения сдачи, если это будет применимо, и т. д.) соподписанту.
        \item {\em Создание частичной транзакции}: соподписант проверяет правильность всей ин\-формации, содержащейся в запущенной транзакции. Он определяет получателей выходов и соответствующие суммы (они могут быть частично основаны на рекомен\-дациях инициатора), входы, которые будут использоваться вместе с соответствую\-щими ложными участниками кольца, и комиссию за проведение транзакции. Он генерирует приватный ключ транзакции (или ключи, если используются подад\-реса), создаёт одноразовые адреса, обязательства по выходам, зашифрованные суммы, маски обязательств по псевдовыходам, и начальные значения обязательств по нулевой сумме. Чтобы доказать, что суммы находятся в пределах допустимого диапазона, он создаёт доказательства диапазона Bulletproofs для всех выходов. Он также генерирует начальные значения для своей подписи (но не даёт по ним обязательств), случайные скалярные величины для подписей MLSAG, частичные образы ключей для имеющихся у него выходов и доказательства легитимности этих частичных образов. Всё это отправляется инициатору транзакции.
        \item {\em Частичная подпись инициатора}: инициатор проверяет, является ли информация в частичной транзакции действительной и соответствует ли его ожиданиям (напри\-мер, суммы и получатели те же, что и должны быть). Он заполняет подписи MLSAG и подписывает их своими приватными ключами, а затем отправляет частично подписанную транзакцию соподписанту вместе со своими раскрытыми начальными значениями.
        \item {\em Завершение создания транзакции}: соподписант завершает подписание транзакции и отправляет её в сеть.
    \end{enumerate}{}
\end{enumerate}{}

\subsubsection*{Подписание с одним обязательством}

В отличие от того, что было рекомендовано нами в главе, посвящённой мультиподписям, для каждой частичной транзакции (инициатором транзакции) даётся только одно обязательство, и оно раскрывается после того, как соподписант уже точно отправит своё начальное значение. Цель обязательств по начальным значениям (например, $\alpha G$) состоит в том, чтобы помешать злоумышленнику использовать своё собственное начальное значение в попытке повлиять на создаваемый запрос, что потенциально позволило бы такому злоумышленнику обнаружить агрегированные ключи других соподписантов (см. подпункт \ref{sec:threshold-schnorr}). Если в момент, когда злоумышленник будет генерировать собственное начальное значение, хотя бы одно частичное начальное значение будет недоступно, то он не сможет (или по крайней мере сможет, но очень малой вероятностью) контролировать запрос.\footnote{По этой же причине инициатор раскрывает свои начальные значения только после определения всей информации транзакции. Поэтому ни один из подписантов не может изменить сообщение MLSAG и повлиять на запрос.}\footnote{Подписание с использованием единственного обязательства можно обобщить как подписание с (M-1)-обязательством, когда только автор частичной транзакции не производит обязательства и раскрытия, а другие соподписанты совершают раскрытие только после того, как информация транзакции будет определена полностью. Например, предположим, что есть адрес, построенный по схеме «3 из 3» соподписантами (A, B, C), которые попытаются подписаться с использованием одного обязательства. Подписанты B и C находятся в сговоре против A, при этом C является инициатором, а B — автором частичной транзакции. C инициирует транзакцию выдачей обязательства, а затем A выдаёт своё начальное значение (без обязательства). Когда B создаёт частичную транзакцию, он может вступить в сговор с C, чтобы контролировать запрос подписи в целях раскрытия приватного ключа A. Также следует обратить внимание на то, что подписание с (M-1)-обязательством — это оригинальная концепция, которая впервые предлагается именно в этой работе, и она не подкреплена какими-либо материалами передовых исследований или уже реализованным кодом. И эта концепция может оказаться совершенно ошибочной.}\footnote{Как вариант, можно рассматривать значение и цель «обязательства» (см. подпункт \ref{sec:commitments}). Как вариант, можно рассматривать значение и цель «обязательства» (см. подпункт 5.1). Как только Элис создаст обязательство по значению A, она «застрянет» на этом и не сможет воспользоваться новой информацией, получаемой в результате события B (инициированного Бобом), которое происходит позже. Более того, если A не было раскрыто, оно не может повлиять на B. Элис и Боб могут быть уверены, что A и B не зависят друг от друга. Мы утверждаем, что подписание с использованием единственного обязательства, как было описано выше, соответствует этому стандарту и эквивалентно подписанию с полным набором обязательств. Если обязательство $c$ является односторонней функцией исходных значений $\alpha_A G$ и $\alpha_B G$ (например, $c = \mathcal{H}_n(\alpha_A G,\alpha_B G)$), то если обязательство по $\alpha_A G$ даётся изначально, $\alpha_B G$ раскрывается после появления $C(\alpha_A G)$, а $\alpha_A G$ раскрывается после появления $\alpha_B G$, то $\alpha_B G$ и $\alpha_A G$ независимы, а $c$ будет случайным как с точки зрения Элис, так и с точки зрения Боба (если они не будут взаимодействовать, или же с самой ничтожной вероятностью).}

Подобное упрощение позволяет убрать один раунд обмена данными, что имеет важные по\-следствия с точки зрения взаимодействия покупателя и продавца.


\subsection{Пользовательский опыт взаимодействия с эскроу-сервисами}
\label{subsec:escrowed-marketplace-escrow-user-experience}

В данном подпункте подробное рассматривается процесс взаимодействия покупателя, про\-давца и модератора при совершении онлайн-покупки за Monero с использованием мультипод\-писи, построенной по схеме «2 из 3».
\begin{enumerate}
    \item {\em Покупатель совершает покупку}
    \begin{enumerate}
        \item {\em Новый сеанс со стороны покупателя}: покупатель входит на торговую онлайн-площадку, и его клиент генерирует новый подадрес, который будет использоваться, если покупатель начнёт оформление нового заказа на покупку.\footnote{Использование нового подадреса для каждого заказа или даже нового подадреса для каждого отдельного продавца или товара такого продавца затрудняет отслеживание продавцами поведения их заказчиков. Это также помогает убедиться в том, что каждый заказ товара уникален, в случае если кто-то покупает одно и то же дважды.} На этой площадке он найдёт необходимых ему продавцов, и каждый продавец предложит на выбор свой товар и соответствующую цену. Каждый продукт имеет базовый ключ, ис\-пользуемый при совершении покупок по схеме multisig, и этот ключ невидим для самого покупателя, но видим для его клиента (то есть программного обеспечения, которое он использует для совершения покупок). Наряду с этим базовым ключом также существует список предварительно выбранных модераторов, и каждый такой модератор имеет базовый ключ и предварительно вычисленный общий секретный публичный ключ, известный продавцу и модератору.\footnote{Самым простым было бы продавцам также включить невидимые для покупателей обязательства по раскрытию значений для транзакций. Однако, чтобы обработать несколько заказов на покупку одного и того же товара, продавцу пришлось бы заранее предоставить множество обязательств для каждого потенциального покупателя. Можно только представить, насколько запутанным мог бы стать весь процесс. Это отчасти дополняет практичность, обеспечиваемую нашим упрощением подписания с использованием одного обязательства.}
        \item {\em Покупатель добавляет товар в корзину}: наш покупатель решает, что он что-то готов купить, выбирает вариант оплаты (например, прямой платеж с использова\-нием мультиподписи по схеме «1 из 2» или мультиподписи по схеме «2 из 3»), и, если он выбирает схему multisig «2 из 3», ему предоставляется список доступных модераторов, из которого он может выбрать нужного. После того как покупатель добавит товар в свою корзину, клиент, невидимый для него (при условии, что была выбрана схема мультиподписи «2 из 3»), использует базовый ключ товара, базовый ключ модератора и общий секретный публичный ключ продавца и модератора, чтобы в сочетании с ключом траты собственного подадреса, созданного для этого сеанса (в качестве базового ключа), построить multisig-адрес покупателя-продавца-модератора по схеме «2 из 3».\footnote{То, как именно должна быть реализована торговая площадка, является предметом свободного решения, поскольку, например, выбор типа оплаты за товар может быть предоставлен пользователю при окончательном оформлении заказа, а не в интерфейсе «добавить в корзину».}\\

        Ключ просмотра является хешем общего секретного приватного ключа покупателя и продавца (а не агрегированным приватным ключом, то есть ключом, созданным до предварительного {\tt premerge (объединения)}), а ключ шифрования для обмена данными между покупателем и продавцом — это хеш ключа просмотра.\footnote{Тот же самый процесс будет использоваться в случае со схемой «1 из 2» без участия модератора.}
        \item {\em Покупатель переходит к оформлению заказа}: покупатель просматривает корзину со всеми товарами и решает перейти к оформлению заказа. Именно на этом этапе он делает доступными свои средства ещё до завершения выполнения заказа. Его клиент строит транзакцию (но пока не подписывает её), которая будет либо прямым платежом продавцу, либо переводом средств на multisig-адрес (с небольшой допла\-той за будущие комиссии). Если происходит перевод средств на multisig-адрес по схеме «2 из 3», клиент также инициирует две транзакции, отправляя деньги с этого адреса. Один может использоваться для оплаты продавцу, а другой - для возмещения покупателю. Частичные образы ключей основаны на транзакции перевода средств, которая еще не была подписана.\\

        На самом деле, ему требуются только исходные значения с обязательствами для двух транзакций, а затем по отдельности одна копия частичных образов ключей (с доказательством легитимности) и одна копия подадреса для данного сеанса. Этот подадрес имеет двойное назначение: это адрес покупателя для возврата сфредств или получения выходов сдачи, а его ключ траты является базовым multisig-ключом покупателя.\footnote{Важно создавать отдельные транзакции, поскольку исходные значения с обязательствами можно использовать только один раз.}
        \item {\em Покупатель решается на проведение оплаты}: после просмотра всех деталей заказа на товар покупатель разрешает оплатить его.\footnote{Если покупатель отменяет заказ на приобретение товара, его транзакция с переводом средств и частичные multisig-транзакции удаляются.} Его клиент завершает подписание транзакции перевода средств и отправляет её в сеть.\footnote{Если в его корзине находились товары от нескольких продавцов, его клиент может создать несколько заказов на приобретение товара и обрабатывать их по отдельности. Все продавцы могут получать оплату посредством одной и той же транзакции с переводом средств.} Он отправляет продавцу заказ на товар вместе с хешем транзакции перевода средств и общим секретным публичным ключом покупателя-модератора, а уже затем инициирует multisig-тран\-закции.\footnote{Клиент покупателя должен отслеживать детали платёжного поручения, такие как итоговая цена, чтобы впоследствии проверить содержание multisig-транзакций перед их подписанием.}
    \end{enumerate}{}
    \item {\em Продавец выполняет заказ на товар}
    \begin{enumerate}
        \item {\em Продавец оценивает заказ на товар}: продавец изучает заказ на товар нашего покупателя, а затем утверждает его для отправки товара. Если ему заплатили напрямую, то рассматривать больше нечего, а если ему заплатили с использованием схемы multisig «1 из 2», то он может совершить транзакцию, заплатив себе с соответствующего адреса. В случае со схемой multisig «2 из 3» его клиент генериру\-ет подадрес для получения суммы, указанной в заказе на товар, и производит две частичные транзакции из транзакций, инициированных покупателем. В транзак\-ции платежа продавцу отправляется сумма, составляющая цену товара, а остальная часть отправляется покупателю в качестве сдачи, в то время как транзакция возме\-щения просто очищает multisig-адрес для покупателя.\footnote{Частичные транзакции могут содержать много общих значений, поскольку они используют одни и те же входы, и только одна из них в конечном итоге должна быть подписана. Мы считаем, что из соображений сохранения модульности и надёжности конструкции лучше обрабатывать их по отдельности.} Обратите внимание, что он реконструирует multisig-адрес на основе базовых ключей покупателя-поставщика-модератора в сочетании с общим секретным публичным ключом покупателя-моде\-ратора.
        \item {\em Продавец отправляет товар}: продавец отправляет товар, а также отправляет покупателю уведомление о выполнении заказа. Это уведомление включает в себя квитанцию о покупке, а также запрос о завершении платежа заказчиком (с этого момента всё относится к схеме multisig «2 из 3»). Подадрес заказа на товар продавца скрыт от пользователя, и он может использоваться в случае необходимости в разре\-шении спора, а также для построения обеих частичных транзакций.
    \end{enumerate}{}
    \item {\em Покупатель завершает проведение платежа или запрашивает возврат}: покупатель может сделать это, как только получит уведомление о выполнении заказа, или может дождаться доставки товара.
    \begin{enumerate}
        \item {\em Покупатель отправляет частично подписанную транзакцию}: покупатель решает, платить ли ему за свою покупку или запросить возврат средств. Его клиент создаёт частичную подпись для соответствующей частичной транзакции и отправляет её продавцу. Предполагается, что любой запрос возврата средств будет содержать объяснение, оправдывающее такой возврат.
        \item {\em Продавец завершает транзакцию}: продавец получает частично подписанную тран\-закцию, завершает её подписание и отправляет в сеть. При необходимости он отправляет покупателю уведомление о возврате средств с доказательством.
    \end{enumerate}{}
    \item {\em Модерируемый спор}: в любой момент после того, как покупатель отправит заказ на приобретение товара, и до того, как с multisig-адресов будут выведены средства, либо продавец, либо покупатель могут решить привлечь модератора к решению возможного спора. Сторона A — это тот, кто начал спор, а Сторона B — ответчик.\footnote{Наша схема разрешения споров предполагает, что действующие лица будут действовать добросовестно. Люди, которые отказываются сотрудничать и, например, не инициируют, и не подписывают сделки, которые не выгодны им самим, без сомнения, сделают процесс намного более утомительным для всех участвующих сторон.}
    \begin{enumerate}
        \item {\em Сторона \_A связывается с модератором}: сторона\_A создаёт две транзакции для оплаты или возврата средств, на этот раз предназначенных для подписания Стороной\_A-модератором, отправляет их модератору вместе с необходимой для создания multisig-адреса информацией (базовые ключи, общий секретный публич\-ный ключ Стороны\_A-Стороны\_B и приватный ключ просмотра) и считывает баланс multisig-кошелька (частичные образы ключей и их доказательства).
        \item {\em Модератор занимается разрешением спора}
        \begin{enumerate}
            \item {\em Модератор признаёт наличие спорной ситуации}: модератор признаёт, что занимается рассмотрением спорной ситуации, и в то же время создаёт частич\-ные транзакции из уже созданных транзакций и отправляет их Стороне\_A. Он обязательно уведомляет Сторону\_B о наличии спорной ситуации, а также создаёт ещё две транзакции со Стороной\_B, чтобы предупредить невыполнение Стороной\_A окончательного решения.
            \item {\em Модератор выносит решение}: модератор просматривает доступные доказатель\-ства и может взаимодействовать со сторонами в случае необходимости сбора дополнительной информации. Он может попытаться и выступить посредником в разрешении спора в надежде, что обе стороны разрешат его без необходимости в вынесении решения.
            \item {\em Спор подходит к концу}: либо покупатель и продавец разрешают спорную ситуацию самостоятельно, либо модератор выносит своё решение, которое сооб\-щает обеим сторонам.
        \end{enumerate}{}
        \begin{itemize}
            \item Примечание: если сторона-ответчик, согласно вынесенному решению, должна получить средства, но по какой-либо причине не предоставила адрес, модератор может попытаться связаться и сотрудничать с ней, чтобы она получила эти средства. Поскольку сторона, инициировавшая спор, не участвует в данном процессе, такая связь может быть установлена (или продолжена) уже после разрешения спора.
        \end{itemize}{}
        \item {\em Сторона\_A или Сторона\_B соглашается с решением}: если транзакции между Стороной\_A и Стороной\_B не были завершены, это означает, что спор был разре\-шён по решению модератора.
        \begin{enumerate}
            \item {\em Сторона\_A соглашается с решением}
            \begin{enumerate}
                \item Сторона\_A завершает свою частичную подпись в транзакции с решением и отправляет её модератору.
                \item Модератор завершает подпись и отправляет транзакцию в сеть.
            \end{enumerate}{}
            \item {\em Сторона\_B соглашается с решением}
            \begin{enumerate}
                \item Сторона\_B создаёт частичную транзакцию для созданной транзакции с решением модератора и отправляет её модератору. Этот этап может быть выполнен до того, как решение станет окончательным, и в этом случае Сторона\_B создаст частичные транзакции для обоих потенциальных воз\-можных решений.
                \item Модератор частично подписывает данную частичную транзакцию и отправ\-ляет её Стороне\_B.
                \item Сторона\_B завершает подписание транзакции и отправляет её в сеть. Он отправляет хеш транзакции модератору.
            \end{enumerate}{}
        \end{enumerate}{}
        \item {\em Модератор закрывает спор}: модератор кратко излагает суть спора и своего реше\-ния и отправляет отчёт покупателю и продавцу.
    \end{enumerate}{}
\end{enumerate}{}

Установлены четыре ключевых оптимизации технического решения.

\subsubsection*{Предварительно выбранные модераторы}

Заранее выбирая модераторов, продавцы могут создать общий секрет для каждого из них, для каждого из своих товаров, и опубликовать соответствующий публичный ключ с информацией о товаре.\footnote{Важно отметить, что эти multisig-адреса по-прежнему устойчивы к тестированию агрегированных ключей, поскольку общие секреты покупателя неизвестны внешним наблюдателям.} Таким образом, покупатели могут создать полный объединённый multisig-адрес за один шаг, как только решат что-то купить, что полностью соответствует требованиям «офлайн-продажи». Предварительный выбор нескольких модераторов позволяет покупате\-лям выбрать того, кому они больше доверяют.

Покупатели, если они не доверяют модераторам, выбранным продавцом, также могут взаимо\-действовать с выбранным уже ими онлайн-модератором, создавая multisig-адрес с базовым ключом товара, предлагаемого продавцом. После получения заказа на приобретение товара продавец может принять нового модератора или отменить продажу.\footnote{Из соображений удобства эскроу-сервис может «всегда быть онлайн», и вместо использования предварительно выбранных модераторов все multisig-адреса «2 из 3» могут создаваться таким сервисом при оформлении заказа на приобретение товара. Ещё вариант — использовать вложенную мультиподпись (см. подпункт \ref{sec:general-key-families}), где предварительно выбранный модератор на самом деле является multisig-группой, созданной по схеме «1 из N». Таким образом, всякий раз при возникновении спора любой модератор из этой группы, который окажется доступным, сможет вмешаться. Для реализации такого варианта, вероятно, потребуются значительные усилия с точки зрения разработки.}

Мы ожидаем от покупателей и продавцов взаимного стремления к выбору хороших модерато\-ров, чтобы со временем создать иерархию модераторов, организованную по качеству и спра\-ведливости предоставляемых ими услуг. Модераторы с более низким качеством предоставляе\-мых услуг или модераторы с более низкой репутацией, скорее всего, будут зарабатывать меньше денег или обслуживать меньшее количество клиентов, или будут заниматься менее значимыми транзакциями.\footnote{Нам пока неясно, какой или способ передачи средств будет наилучшим для модераторов или наиболее вероятно будет использоваться ими. Возможно, им будет выплачиваться фиксированная или процентная ставка за каждую модерируемую транзакцию или транзакцию, в которую они будут добавлены в качестве модератора (а затем, если комиссия не была предусмотрена в исходных частичных транзакциях, они будут просто отказываться от участия в разрешения спора) или же пользователи и/или поставщики, и/или торговые площадки будут заключать с ними договор.}

\subsubsection*{Подадреса и идентификаторы товаров}

Продавцы создают новый базовый ключ для каждой линейки товаров / каждого идентифика\-тора, и эти ключи используются для создания multisig-адресов, построенных по схеме «2 из 3».\footnote{Этот базовый ключ также используется для покупки с использованием мультиподписей, построенных по схеме «1 из 2». Мы считаем важным не раскрывать приватный ключ траты в канале обмена данными, поэтому использование общего секрета покупателя и продавца в данном случае имеет вполне определённый смысл.} Когда продавцы выполняют заказ, они создают уникальный подадрес для получения средств, который также можно использовать для сопоставления заказов с полученными пла\-тежами.

Здесь эффективно выполняется требование к «платежам на основе заказа на приобретение товара», в частности потому, что средства, направляемые на разные подадреса, будут легко доступны из одного и того же кошелька (см. подпункт \ref{sec:subaddresses}).

\subsubsection*{Предварительные частичные транзакции}

Транзакции с мультиподписями реализуются в несколько раундов, поэтому мы начинаем их проводить как можно скорее. Для удобства пользователя частичные транзакции, которые используются редко (например, транзакции возмещения), выполняются раньше, поэтому они сразу же доступны для подписания, если в них возникает необходимость.

\subsubsection*{Условный доступ модератора}

Как для повышения эффективности, так и для повышения уровня анонимности модераторам требуется доступ к деталям сделки исключительно при урегулировании споров. Для дости\-жения этой цели мы делаем приватный multisig-ключ просмотра хешем общего секретного приватного ключа покупателя-продавца $k^{v,grp}_{purchase-order} = \mathcal{H}_n(T_{mv},k^{sh,\textrm{2x3}}_{AB})$, где $T_{mv}$ является разделителем домена ключа просмотра торговой площадки, A и B обозначают продавца и покупателя, соответственно, а $k^{sh,\textrm{2x3}}_{AB} = \mathcal{H}_n(k^{base}_{A}*k^{base}_{B} G)$. Мы предполагаем то, что он захочет «взглянуть» на скрипт обмена данными между покупателем и продавцом. Другими словами, ключом шифрования обмена данными является $k^{ce}_{purchase-order} = \mathcal{H}_n(T_{me},k^{v,grp}_{purchase-order})$ (где $T_{me}$ является разделителем домена ключа шифрования торговой площадки).\footnote{Разделение ключа просмотра и ключа шифрования позволяет давать права только на просмотр журнала обмена данными без права просмотра истории транзакций конкретного multisig-адреса.}\footnote{Этот метод также используется в случае с multisig-адресами. созданными по схеме «1 из 2».}

Модераторы получают доступ к обмену данными между покупателем и продавцом и возмож\-ность подтверждать платежи только тогда, когда одна из первоначальных сторон выдаёт им ключ просмотра.\footnote{Модераторам важно убедиться в том, что журнал обмена данными, который они получают, не был подделан. Одним из способов является включение каждым соподписантом подписанного хеша текущего журнала обмена данными при отправлении нового сообщения. В этом случае модераторы смогут просмотреть ряд хешированных журналов и выявить возможные расхождения. Это также помогло бы соподписантам выявить сообщения, которые не были переданы, и в качестве альтернативы создать доказательства того, что конкретные подписанты действительно получили определённые сообщения.}

Кроме того, продавцы могут проверить, не является ли хост торговой площадки (который также может быть доступен только для модераторов в зависимости от того, как реализуется данная концепция) MITM («атакой посредника») их разговоров с заказчиками (то есть не притворяется ли он покупателем или продавцом). Это делается путём проверки соответствия базовых ключей, публикуемых ими для каждого товара, тому, что отображается. Поскольку базовый ключ покупателя, который используется для создания multisig-адреса, также явля\-ется частью ключа шифрования, вредоносному хосту будет трудно организовать MITM-атаку.