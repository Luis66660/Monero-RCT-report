\begin{appendices}

\renewcommand{\theFancyVerbLine}{%
	\textcolor{red}{\small
		\arabic{FancyVerbLine}}}

\chapter{Структура транзакций {\tt RCTTypeBulletproof2}}
\label{appendix:RCTTypeBulletproof2}

В этом приложении мы приводим пример распечатанных данных реальной транзакции Monero типа {\tt RCTTypeBulletproof2}, а также свои пояснения к соответствующим полям.

Данные были получены при помощи блок-эксплорера \url{https://xmrchain.net}. Также их мож\-но извлечь, воспользовавшись командой print tx {\tt print\_tx <TransactionID> +hex +json} в демон-программе {\tt monerod} в неотключённом режиме. {\tt <TransactionID>} является хешем тран\-закции (см. подпункт \ref{subsec:transaction-id}). Первая распечатанная строка показывает фактический запуск команды.% see chapter 7 blockchain on TransactionID

Чтобы воспроизвести наши результаты, пользователи могут сделать следующее:%\footnote{Blockchain data can also be found with a web block explorer, such as \url{https://moneroblocks.info/} or \url{https://xmrchain.net}.}
\begin{enumerate}
    \item Вам понадобится инструмент с поддержкой командной строки (CLI) Monero, который можно скачать на \url{https://web.getmonero.org/downloads/} (и не только). Следует вы\-брать Command Line Tools Only (только инструменты с поддержкой командной строки) для вашей операционной системы, переместить файл в нужное место и разархивировать его.
    \item После этого следует открыть terminal/command line (командная строка) и перейти к папке, созданной в результате разархивирования.
    \item Используя {\tt monerod}, открыть демон-программу {\tt ./monerod}. Начнётся загрузка блокчейна Monero. К сожалению, на данный момент простого способа распечатки транзакций на вашей системе не существует (например, без использования блокчейн-эксплорера), поэтому приходится загружать блокчейн.
    \item После того как процесс синхронизации будет завершён, можно будет использовать такие команды, как {\tt print\_tx}. Чтобы узнать другие команды, воспользуйтесь разделом {\tt help} (помощь).
\end{enumerate}

%For editorial reasons we have shortened long hexadecimal chains, presenting only the beginning and end as in {\tt 0200010c7f[...]409}.

Компонент {\tt rctsig\_prunable}, как следует из его названия, теоретически можно {\sl вырезать} из блокчейна. То есть после того, как блок будет согласован, а существующие правила, определяющие длину блокчейна, исключают всяческую возможность атаки путём двойной траты, всё это поле можно будет вырезать и заменить его хешем в дереве Меркла.

Образы ключей хранятся отдельно в той части транзакций, которая не может быть удалена. Эти компоненты важны с точки зрения обнаружения атак путём двойной траты и не могут быть вырезаны.
\\

В нашем примере транзакция имеет 2 входа и 2 выхода и была добавлена в блокчейн с временной меткой 2020-03-02 19:01:10 UTC (как было указано майнером блока).

\begin{Verbatim}[commandchars=\\\{\}, numbers=left]
print_tx 84799c2fc4c18188102041a74cef79486181df96478b717e8703512c7f7f3349
Found in blockchain at height 2045821
\{
  "version": 2, 
  "unlock_time": 0, 
  "vin": [ \{
      "key": \{
        "amount": 0, 
        "key_offsets": [ 14401866, 142824, 615514, 18703, 5949, 22840, 5572, 16439,
        983, 4050, 320
        ], 
        "k_image": "c439b9f0da76ca0bb17920ca1f1f3f1d216090751752b091bef9006918cb3db4"
      \}
    \}, \{
      "key": \{
        "amount": 0, 
        "key_offsets": [ 14515357, 640505, 8794, 1246, 20300, 18577, 17108, 9824, 581,
        637, 1023
        ], 
        "k_image": "03750c4b23e5be486e62608443151fa63992236910c41fa0c4a0a938bc6f5a37"
      \}
    \}
  ], 
  "vout": [ \{
      "amount": 0, 
      "target": \{
        "key": "d890ba9ebfa1b44d0bd945126ad29a29d8975e7247189e5076c19fa7e3a8cb00"
      \}
    \}, \{
      "amount": 0, 
      "target": \{
        "key": "dbec330f8a67124860a9bfb86b66db18854986bd540e710365ad6079c8a1c7b0"
      \}
    \}
  ], 
  "extra": [ 1, 3, 39, 58, 185, 169, 82, 229, 226, 22, 101, 230, 254, 20, 143,
  37, 139, 28, 114, 77, 160, 229, 250, 107, 73, 105, 64, 208, 154, 182, 158, 200,
  73, 2, 9, 1, 12, 76, 161, 40, 250, 50, 135, 231
  ], 
  "rct_signatures": \{
    "type": 4, 
    "txnFee": 32460000, 
    "ecdhInfo": [ \{
        "amount": "171f967524e29632"
      \}, \{
        "amount": "5c2a1a9f54ccf40b"
      \}], 
    "outPk": [ "fed8aded6914f789b63c37f9d2eb5ee77149e1aa4700a482aea53f82177b3b41",
    "670e086e40511a279e0e4be89c9417b4767251c5a68b4fc3deb80fdae7269c17"]
  \}, 
  "rctsig_prunable": \{
    "nbp": 1, 
    "bp": [ \{
        "A": "98e5f23484e97bb5b2d453505db79caadf20dc2b69dd3f2b3dbf2a53ca280216", 
        "S": "b791d4bc6a4d71de5a79673ed4a5487a184122321ede0b7341bc3fdc0915a796", 
        "T1": "5d58cfa9b69ecdb2375647729e34e24ce5eb996b5275aa93f9871259f3a1aecd", 
        "T2": "1101994fea209b71a2aa25586e429c4c0f440067e2b197469aa1a9a1512f84b7", 
        "taux": "b0ad39da006404ccacee7f6d4658cf17e0f42419c284bdca03c0250303706c03", 
        "mu": "cacd7ca5404afa28e7c39918d9f80b7fe5e572a92a10696186d029b4923fa200", 
        "L": [ "d06404fc35a60c6c47a04e2e43435cb030267134847f7a49831a61f82307fc32",
        "c9a5932468839ee0cda1aa2815f156746d4dce79dab3013f4c9946fce6b69eff",
        "efdae043dcedb79512581480d80871c51e063fe9b7a5451829f7a7824bcc5a0b",
        "56fd2e74ac6e1766cfd56c8303a90c68165a6b0855fae1d5b51a2e035f333a1d",
        "81736ed768f57e7f8d440b4b18228d348dce1eca68f969e75fab458f44174c99",
        "695711950e076f54cf24ad4408d309c1873d0f4bf40c449ef28d577ba74dd86d",
        "4dc3147619a6c9401fec004652df290800069b776fe31b3c5cf98f64eb13ef2c"
        ], 
        "R": [ "7650b8da45c705496c26136b4c1104a8da601ea761df8bba07f1249495d8f1ce",
        "e87789fbe99a44554871fcf811723ee350cba40276ad5f1696a62d91a4363fa6",
        "ebf663fe9bb580f0154d52ce2a6dae544e7f6fb2d3808531b0b0749f5152ddbf",
        "5a4152682a1e812b196a265a6ba02e3647a6bd456b7987adff288c5b0b556ec6",
        "dc0dcb2e696e11e4b62c20b6bfcb6182290748c5de254d64bf7f9e3c38fb46c9",
        "101e2271ced03b229b88228d74b36088b40c88f26db8b1f9935b85fb3ab96043",
        "b14aae1d35c9b176ac526c23f31b044559da75cf95bc640d1005bfcc6367040b"
        ], 
        "a": "4809857de0bd6becdb64b85e9dfbf6085743a8496006b72ceb81e01080965003", 
        "b": "791d8dc3a4ddde5ba2416546127eb194918839ced3dea7399f9c36a17f36150e", 
        "t": "aace86a7a1cbdec3691859fa07fdc83eed9ca84b8a064ca3f0149e7d6b721c03"
      \}
    ], 
    "MGs": [ \{
        "ss": [[ "d7a9b87cfa74ad5322eedd1bff4c4dca08bcff6f8578a29a8bc4ad6789dee106",
        "f08e5dfade29d2e60e981cb561d749ea96ddc7e6855f76f9b807842d1a17fe00"],
        ["de0a86d12be2426f605a5183446e3323275fe744f52fb439041ad2d59136ea0b",
        "0028f97976630406e12c54094cbbe23a23fe5098f43bcae37339bfc0c4c74c07"],
        ["f6eef1f99e605372cb1ec2b3dd4c6e56a550fec071c8b1d830b9fda34de5cc05",
        "cd98fc987374a0ac993edf4c9af0a6f2d5b054f2af601b612ea118f405303306"],
        ["5a8437575dae7e2183a1c620efbce655f3d6dc31e64c96276f04976243461e08",
        "5090103f7f73a33024fbda999cd841b99b87c45fa32c4097cdc222fa3d7e9502"],
        ["88d34246afbccbd24d2af2ba29d835813634e619912ea4ca194a32281ac14e0e",
        "eacdf59478f132dd8dbb9580546f96de194092558ffceeff410ee9eb30ce570e"],
        ["571dab8557921bbae30bda9b7e613c8a0cff378d1ec6413f59e4972f30f2470d",
        "5ca78da9a129619299304d9b03186233370023debfdaddcd49c1a338c1f0c50d"],
        ["ac8dbe6bb28839cf98f02908bd1451742a10c713fdd51319f2d42a58bf1d7507",
        "7347bf16cba5ee6a6f2d4f6a59d1ed0c1a43060c3a235531e7f1a75cd8c8530d"],
        ["b8876bd3a5766150f0fbc675ba9c774c2851c04afc4de0b17d3ac4b6de617402",
        "e39f1d2452d76521cbf02b85a6b626eeb5994f6f28ce5cf81adc0ff2b8adb907"],
        ["1309f8ead30b7be8d0c5932743b343ef6c0001cef3a4101eae98ffde53f46300",
        "370693fa86838984e9a7232bca42fd3d6c0c2119d44471d61eee5233ba53c20f"],
        ["80bc2da5fc5951f2c7406fce37a7aa72ffef9cfa21595b1b68dfab4b7b9f9f0c",
        "c37137898234f00bce00746b131790f3223f97960eefe67231eb001092f5510c"],
        ["01c89e07571fd365cac6744b34f1b44e06c1c31cbf3ee4156d08309345fdb20e",
        "a35c8786695a86c0a4e677b102197a11dadc7171dd8c2e1de90d828f050ec00f"]], 
        "cc": "0d8b70c600c67714f3e9a0480f1ffc7477023c793752c1152d5df0813f75ff0f"
      \}, \{
        "ss": [[ "4536e585af58688b69d932ef3436947a13d2908755d1c644ca9d6a978f0f0206",
        "9aab6509f4650482529219a805ee09cd96bb439ee1766ced5d3877bf1518370b"],
        ["5849d6bf0f850fcee7acbef74bd7f02f77ecfaaa16a872f52479ebd27339760f",
        "96a9ec61486b04201313ac8687eaf281af59af9fd10cf450cb26e9dc8f1ce804"],
        ["7fe5dcc4d3eff02fca4fb4fa0a7299d212cd8cd43ec922d536f21f92c8f93f00",
        "d306a62831b49700ae9daad44fcd00c541b959da32c4049d5bdd49be28d96701"],
        ["2edb125a5670d30f6820c01b04b93dd8ff11f4d82d78e2379fe29d7a68d9c103",
        "753ac25628c0dada7779c6f3f13980dfc5b7518fb5855fd0e7274e3075a3410c"],
        ["264de632d9cb867e052f95007dfdf5a199975136c907f1d6ad73061938f49c01",
        "dd7eb6028d0695411f647058f75c42c67660f10e265c83d024c4199bed073d01"],
        ["b2ac07539336954f2e9b9cba298d4e1faa98e13e7039f7ae4234ac801641340f",
        "69e130422516b82b456927b64fe02732a3f12b5ee00c7786fe2a381325bf3004"],
        ["49ea699ca8cf2656d69020492cdfa69815fb69145e8f922bb32e358c23cebb0f",
        "c5706f903c04c7bed9c74844f8e24521b01bc07b8dbf597621cceeeb3afc1d0c"],
        ["a1faf85aa942ba30b9f0511141fcab3218c00953d046680d36e09c35c04be905",
        "7b6b1b6fb23e0ee5ea43c2498ea60f4fcf62f70c7e0e905eb4d9afa1d0a18800"],
        ["785d0993a70f1c2f0ac33c1f7632d64e34dd730d1d8a2fb0606f5770ed633506",
        "e12777c49ffc3f6c35d27a9ccb3d9b8fed7f0864a880f7bae7399e334207280e"],
        ["ab31972bf1d2f904d6b0bf18f4664fa2b16a1fb2644cd4e6278b63ade87b6d09",
        "1efb04fe9a75c01a0fe291d0ae00c716e18c64199c1716a086dd6e32f63e0a07"],
        ["a6f4e21a27bf8d28fc81c873f63f8d78e017666adbf038da0b83c2ad04ef6805",
        "c02103455f93c2d7ec4b7152db7de00d1c9e806b1945426b6773026b4a85dd03"]], 
        "cc": "d5ac037bb78db41cf924af713b7379c39a4e13901d3eac017238550a1a3b910a"
      \}],
    "pseudoOuts": [ "b313c1ae9ca06213684fbdefa9412f4966ad192bc0b2f74ed1731381adb7ab58",
    "7148e7ef5cfd156c62a6e285e5712f8ef123575499ff9a11f838289870522423"]
  \}
\}
\end{Verbatim}



\section*{Компоненты транзакции}
	
\begin{itemize}
    \item (строка 2) - команда {\tt print\_tx} указывает блок, в котором была найдена транзакция, что мы и воспроизводим здесь в демонстрационных целях.
	\item {\tt version} (строка 4) - формат транзакции / версия по времени создания; «2» соответствует протоколу RingCT.
	\item {\tt unlock\_time} (строка 5) - не позволяет тратить выходы транзакции до истечения опреде\-лённого срока. Это может быть либо высота блока, либо временная метка UNIX, если высота больше, чем начало времени UNIX. По умолчанию устанавливается нулевое значение, но предельного значения не задаётся.	
	\item {\tt vin} (строки 6-23) - список входов (в данном случае их два).
	\item {\tt amount} (строка 8) - обрезанное (унаследованное) поле суммы, используемое для транзак\-ций первого типа.
	\item {\tt key\_offset} (строка 9) - позволяет верификаторам находить ключи и обязательства участников кольца в блокчейне и убедиться в том, что эти участники являются легитим\-ными. Первый офсет является абсолютным в истории блокчейна, а каждый последую\-щий офсет будет относительным предшествующему. Например, при реальных офсетах \{7,11,15,20\} в блокчейн записываются \{7,4,4,5\}. Верификаторы вычисляют последний офсет как (7+4+4+5 = 20) (см. подпункт \ref{subsec:space-and-ver-rcttypefull}).
	\item {\tt k\_image} (строка 12) - образ ключа \(\tilde{K_j}\) из подпункта \ref{sec:MLSAG}, где $j = 1$, поскольку это первый вход.
	\item {\tt vout} (строки 24-35) - список выходов (в данном случае их два).
	\item {\tt amount} (строка 25) - обрезанное поле суммы, используемое для транзакций первого типа.
	\item {\tt key} (строка 27) - одноразовый ключ адресата для выхода $t = 0$, как было описано в подпункте \ref{sec:one-time-addresses}
	\item {\tt extra} (строки 36-39) - дополнительные\marginnote{src/crypto- note\_basic/ tx\_extra.h} данные, включающие {\em в себя публичный ключ транзакции}, то есть общий секрет $r G$ (см. подпункт \ref{sec:one-time-addresses}) и ID платежа или зашифрован\-ный ID платежа (см. подпункт \ref{sec:integrated-addresses}. Как правило, это работает следующим образом: каждое число имеет размер один байт (может находиться в диапазоне от 0 до 255) и каждое может быть в поле «тег» или «размер». В теге указывается, какая информация будет следующей, а размер указывает, сколько байт эта информация занимает. Первое число всегда будет тегом. В данном случае «1» означает «публичный ключ транзакции». Размер публичных ключей транзакции всегда составляет 32 байта, поэтому нет никакой необходимости указывать их размер. Через 32 числа мы находим новый тег «2», озна\-чающий «дополнительный нонс», длина которого равна «9», а следующий байт будет иметь значение «1», что обозначает 8-байтовый зашифрованный идентификатор плате\-жа (в дополнительном нонсе для различных целей может содержаться до пяти полей). Ещё через восемь байтов поле заканчивается. Более подробная информация содержится в работе \cite{extra-field-stackexchange}. (Примечание: в оригинальной спецификации Cryptonote первый байт указывал на размер поля.  Monero не использует этого.) \cite{tx-extra-field}
	\item {\tt rct\_signatures} (строки 40-50) - первая часть данных подписи.
	\item {\tt type} (строка 41) - тип подписи: {\tt RCTTypeBulletproof2} является четвёртым типом. Обре\-занными типами RingCT были {\tt RCTTypeFull} и {\tt RCTTypeSimple}, 1 и 2 тип, соответственно. Майнинговые транзакции используют нулевой тип подписи, {\tt RCTTypeNull}.
	\item {\tt txnFee} (строка 42) - комиссия за проведение транзакции, указываемая простым текстом. В данном случае размер комиссии составляет 0,00003246 XMR.
	\item {\tt ecdhInfo} (строки 43-47) - «информация по эллиптической кривой Диффи-Хелмана»: скрытая сумма для каждого выхода $t \in \{0, ..., p-1\}$, где $p = 2$.
    \item {\tt amount} (строка 44) - поле {\sl суммы} для $t = 0$, как было указано в подпункте \ref{sec:pedersen_monero}
    \item {\tt outPk} (строки 48-49) - обязательства по каждому выходу (см. подпункт \ref{sec:ringct-introduction}.
    \item {\tt rctsig\_prunable} (строки 51-132) - вторая часть данных подписи.
    \item {\tt nbp} (строка 52) - количество доказательств диапазона Bulletproof в данной транзакции.
    \item {\tt bp} (строки 53-80) - элементы доказательства Bulletproof (доказательства Bulletproofs не рассматриваются в рамках настоящего документа, поэтому мы не вдаёмся в подробно\-сти).\vspace{.175cm}
    \[\Pi_{BP} = (A, S, T_1, T_2, \tau_x, \mu, \mathbb{L}, \mathbb{R}, a, b, t)\]
    \item {\tt MGs} (строки 81-129) - подписи MLSAG.
    \item {\tt ss} (строки 82-103) - компоненты \(r_{i,1}\) и \(r_{i,2}\) подписи MLSAG для первого входа.\vspace{.175cm}
    \[\sigma_j(\mathfrak{m}) = (c_1, r_{1, 1}, r_{1, 2}, ..., r_{v+1, 1}, r_{v+1, 2})\]
    \item {\tt cc} (строка 104) - компонент \(c_1\) вышеупомянутой подписи MLSAG.
    \item {\tt pseudoOuts} (строки 130-131) - обязательства по псевдовыходам $C'^a_j$, о которых говорится в подпункте  \ref{sec:pedersen_monero}. Пожалуйста, не забывайте о том, что сумма этих обязательств будет равна сумме двух обязательств по выходам данной транзакции (плюс обязательство по комиссии за проведение транзакции $f H$).
\end{itemize}




\chapter{Содержание блока}
\label{appendix:block-content}

В этом приложении мы покажем структуру стандартного блока, а именно блока 1582196, если считать от генезис-блока. Блок содержит 5 транзакций и был добавлен в блокчейн с временной меткой 2018-05-27 21:56:01 UTC (как было указано майнером блока).

\begin{Verbatim}[commandchars=\\\{\}, numbers=left]
print_block 1582196
timestamp: 1527458161
previous hash: 30bb9b475a08f2ea6fe07a1fd591ea209a7f633a400b2673b8835a975348b0eb
nonce: 2147489363
is orphan: 0
height: 1582196
depth: 2
hash: 50c8e5e51453c2ab85ef99d817e166540b40ef5fd2ed15ebc863091ca2a04594
difficulty: 51333809600
reward: 4634817937431
\{
  "major_version": 7,
  "minor_version": 7,
  "timestamp": 1527458161,
  "prev_id": "30bb9b475a08f2ea6fe07a1fd591ea209a7f633a400b2673b8835a975348b0eb",
  "nonce": 2147489363,
  "miner_tx": \{
    "version": 2,
    "unlock_time": 1582256,
    "vin": [ \{
        "gen": \{
          "height": 1582196
        \}
      \}
    ],
    "vout": [ \{
        "amount": 4634817937431,
        "target": \{
          "key": "39abd5f1c13dc6644d1c43db68691996bb3cd4a8619a37a227667cf2bf055401"
        \}
      \}
    ],
    "extra": [ 1, 89, 148, 148, 232, 110, 49, 77, 175, 158, 102, 45, 72, 201, 193,
    18, 142, 202, 224, 47, 73, 31, 207, 236, 251, 94, 179, 190, 71, 72, 251, 110, 
    134, 2, 8, 1, 242, 62, 180, 82, 253, 252, 0
    ],
    "rct_signatures": \{
      "type": 0
    \}
  \},
  "tx_hashes": [ "e9620db41b6b4e9ee675f7bfdeb9b9774b92aca0c53219247b8f8c7aecf773ae",
                 "6d31593cd5680b849390f09d7ae70527653abb67d8e7fdca9e0154e5712591bf",
                 "329e9c0caf6c32b0b7bf60d1c03655156bf33c0b09b6a39889c2ff9a24e94a54",
                 "447c77a67adeb5dbf402183bc79201d6d6c2f65841ce95cf03621da5a6bffefc",
                 "90a698b0db89bbb0704a4ffa4179dc149f8f8d01269a85f46ccd7f0007167ee4"
  ]
\}
\end{Verbatim}



\section*{Компоненты блока}

\begin{itemize}
	\item (строки 2-10) - информация блока, собранная программным обеспечением. Но если быть более точными, она фактически не является частью блока.
    \item {\tt is orphan} (строка 5) - указывает на то, что этот блок был «заброшен». Обычно во время форка узлы сохраняют все ветви и отбрасывают их только в том случае, если появляется лидер по совокупной сложности.
    \item {\tt depth} (строка 7) - в копии блокчейна глубиной любого заданного блока называют то, насколько далеко от самого последнего блока в блокчейне находится такой заданный блок.
    \item {\tt hash} (строка 8) - это идентификатор (ID) блока.
    \item {\tt difficulty} (строка 9) - значение сложности не сохраняется в блоке, так как пользовате\-ли могут вычислить сложность {\em всех} блоков на основе временных меток и следуя прави\-лам, которые приводятся в подпункте \ref{sec:difficulty}.
    \item {\tt major\_version} (строка 12) - соответствует версии протокола, используемой для верифи\-кации этого блока.
    \item {\tt minor\_version} (строка 13) - изначально задумывалась как механизм голосования среди майнеров, а теперь просто является повторением {\tt major\_version}. Так как поле не занимает много места, возможно, что разработчики решили не тратить усилий на его удаление.
    \item {\tt timestamp} (строка 14) - числовое выражение временной метки UTC этого блока, указы\-ваемой майнером блока.
    \item {\tt prev\_id} (строка 15) - идентификатор предыдущего блока. В данном случае этим выра\-жается сущность блокчейна Monero.
    \item {\tt nonce} (строка 16) - нонс используется майнером блока для прохождения целевого значе\-ния сложности. Любой может повторно вычислить доказательство работы и проверить, является ли нонс действительным.
    \item {\tt miner\_tx} (строки17-40) - транзакция майнера этого блока.
    \item {\tt version} (строка 18) - формат транзакции / версия по времени создания; «2» соответ\-ствует протоколу RingCT.
    \item {\tt unlock\_time} (строка 19) — выход транзакции майнера не может быть потрачен до блока 1 582 256\nth, пока не будет вычислено ещё 59 блоков (время блокировки составляет 60 блоков, поскольку выход нельзя будет потратить, пока не истечёт время 60 блоков, например, $2*60 = 120$ минут).
    \item {\tt vin} (строки 20-25) — входы транзакции майнера. Здесь нет ни одного, так как транзак\-ция майнера используется для генерирования вознаграждений за вычисление блоков и комиссий за проведение транзакций.
    \item {\tt gen} (строка 21) — сокращение от generate (генерировать).
    \item {\tt height} (строка 22) — высота блока, для которой было сгенерировано вознаграждение за вычисление блока для данной транзакции майнера. Вознаграждение за блок может быть сгенерировано только один раз для каждой высоты блока.
    \item {\tt vout} (строки 26-32) — выходы транзакции майнера.
    \item {\tt amount} (строка 27) — сумма, распределяемая в транзакции майнера, содержащая возна\-граждение за вычисление блока и комиссии за транзакции в этом блоке. Записывается в атомных единицах.
    \item {\tt key} (строка 29) — одноразовый адрес, указывающий принадлежность суммы, указанной в транзакции майнера.
    \item {\tt extra} (строки 33-36) — дополнительная информация транзакции майнера, включающая в себя публичный ключ транзакции.
    \item {\tt type} (строка 38) — тип транзакции (в этом случае это «0» ({\tt RCTTypeNull}), обозначающий транзакцию майнера).
    \item {\tt tx\_hashes} (строки 41-46) — все идентификаторы транзакций, входящих в этот блок (кроме идентификатора транзакции майнера, который будет следующим:\linebreak {\tt 06fb3e1cf889bb972774a8535208d98db164394ef2b14ecfe74814170557e6e9}).
\end{itemize}




\chapter{Генезис-блок}
\label{appendix:genesis-block}

В этом приложении нами будет представлена структура генезис-блока Monero. Блок содержит 0 транзакций (с ним просто было отправлено первое вознаграждение за вычисление блока пользователю thankful\_for\_today \cite{bitmonero-launched}). Основатель Monero не добавлял временной метки. Возможно, это было пережитком Bytecoin, монеты, в результате форка которой получился код Monero, и создатели которой очевидно пытались скрыть внушительный премайнинг \cite{monero-history}, или же скрыто пользовались связанным с криптовалютой программным обеспечением или сервисами \cite{bytecoin-network}.

Блок 1 был добавлен в блокчейн с временной меткой 2014-04-18 10:49:53 UTC (как было указано майнером блока), так что мы можем предположить, что генезис-блок был создан в тот же самый день. Это совпадает с датой запуска, заявленной thankful\_for\_today \cite{bitmonero-launched}.

\begin{Verbatim}[commandchars=\\\{\}, numbers=left]
print_block 0
timestamp: 0
previous hash: 0000000000000000000000000000000000000000000000000000000000000000
nonce: 10000
is orphan: 0
height: 0
depth: 1580975
hash: 418015bb9ae982a1975da7d79277c2705727a56894ba0fb246adaabb1f4632e3
difficulty: 1
reward: 17592186044415
\{
  "major_version": 1,
  "minor_version": 0,
  "timestamp": 0,
  "prev_id": "0000000000000000000000000000000000000000000000000000000000000000",
  "nonce": 10000,
  "miner_tx": \{
    "version": 1,
    "unlock_time": 60,
    "vin": [ \{
        "gen": \{
          "height": 0
        \}
      \}
    ],
    "vout": [ \{
        "amount": 17592186044415,
        "target": \{
          "key": "9b2e4c0281c0b02e7c53291a94d1d0cbff8883f8024f5142ee494ffbbd088071"
        \}
      \}
    ],
    "extra": [ 1, 119, 103, 170, 252, 222, 155, 224, 13, 207, 208, 152, 113, 94, 188, 
    247, 244, 16, 218, 235, 197, 130, 253, 166, 157, 36, 162, 142, 157, 11, 200, 144, 
    209
    ],
    "signatures": [ ]
  \},
  "tx_hashes": [ ]
\}
\end{Verbatim}



\section*{Компоненты генезис-блока}

Так как мы использовали одно и то же программное обеспечение для распечатки генезис-блока и блока, описанного в Приложении \ref{appendix:block-content}, их структура практически одинакова. Но нам бы хотелось отметить некоторые важные отличия.

\begin{itemize}
	\item {\tt difficulty} (строка 9) — значение сложности генезис-блока указывается как 1, что означает, что может использоваться любой нонс.
	\item {\tt timestamp} (строка 14) — у генезис-блока нет значимой временной метки.
	\item {\tt prev\_id} (строка 15) — нами используются пустые 32 байта в качестве идентификатора предыдущего блока по определению.
	\item {\tt nonce} (строка 16) - $n = 10000$ по определению.
	\item {\tt amount} (строка 27) — точно соответствует нашим вычислениям вознаграждения за пер\-вый блок (17,592186044415 XMR), о чём говорится в подпункте \ref{subsec:block-reward}.
	\item {\tt key} (строка 29) — самые первые Monero были переданы основателю thankful\_for\_today.
	\item {\tt extra} (строки 33-36) — благодаря шифрованию, описанному в Приложении \ref{appendix:RCTTypeBulletproof2}, поле {\tt extra} транзакции майнера генезис-блока содержит один публичный ключ транзакции.
	\item {\tt signatures} (строка 37) — в генезис-блоке отсутствуют какие-либо подписи. В данном случае это просто артефакт функции {\tt print\_block}. То же самое относится и к {\tt tx\_hashes} в строке 39.
\end{itemize}


\end{appendices}