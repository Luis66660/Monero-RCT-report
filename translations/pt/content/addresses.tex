\chapter{Endereços}
\label{chapter:addresses}

A posse de moeda digital, guardada na lista de blocos, é definida pela posse de endereços. Somente o proprietário do endereço, pode gastar o valor digital alí contido (enp).
Mais especificamente, para dadas transacções um {\em endereço} possuí certas {\em saídas}.
Ou seja imagine-se o proprietário do endereço, com o direito a gastar valor digital.
Para gastar de uma {\em saída}, o proprietário do endereço (que tem posse dessa saída), faz referência a essa saída como uma {\em entrada} numa nova transacção.
Esta nova transacção têm então novas {\em saídas} controladas por novos endereços.
É importante de notar que o proprietário de uma saída só a pode gastar uma vez.
Numa transacção a soma do valor das {\em entradas} é igual á soma do valor das {\em saídas}. O montante de um endereço é a soma de todos os valores contidos nas {\em saídas} não gastas.
Programas conhecidos como {\em carteiras} são então usados, para encontrar e organizar as saídas controladas por um endereço. Como também para manter a custódia das respectivas chaves privadas, criar novas transacções e submeter estas á rede de Monero. Para que finalmente essa transacção válida seja incluída num bloco.

\section{Chaves}
\label{sec:user-keys}

Um moneriano tem então dois pares de chaves privadas e públicas, \((k^v, K^v)\) e \( (k^s, K^s) \). Estas chaves são geradas como descrito na Secção \ref{ec:keys}.
\marginnote{src/ common/ base58.cpp {\tt encode\_ addr()}}

Um {\em endereço} é o par de chaves públicas \((K^s, K^v)\).\newline Ao qual corresponde o par de chaves privadas \((k^s, k^v)\).\footnote{Para comunicar um endereço a outros utilizadores, é usual codificá-lo na base-58, um esquema de codificação para converter binário para texto. Este esquema foi criado para Bitcoin \cite{base-58-encoding}. Veja-se \cite{luigi-address} para mais detalhes.}
%To communicate an address to other users, it is extremely common to encode it in base-58, a binary-to-text encoding scheme first created for Bitcoin \cite{base-58-encoding}. See \cite{luigi-address} for more details.}

Usar dois conjuntos de chaves permite ter funções distintas.\newline A chave privada $k^s$ é a {\em chave de gasto}.\newline A chave privada $k^v$ é a {\em chave de ver} .  

Com a chave $k^v$ é possível determinar se um endereço possuí uma saída.\newline E com a chave $k^s$, é então possível gastar essa saída numa nova transacção. 
%(and retroactively figure out it has been spent).
\footnote{É actualmente comum para que a chave de ver $k^v$ seja igual a $\mathcal{H}_n(k^s)$. Isto significa que uma pessoa só precisa de guardar a sua chave de gasto $k^s$ para aceder ás saídas que possui (gastas e não gastas). A chave de gasto é típicamente representada como uma série de 25 palavras (em que a última palavra serve como soma de verificação). Outros métodos menos populares incluem: gerar $k^v$ e $k^s$ como números aleatórios separados, ou gerar uma mnemónica aleatória de 12 palavras $a$, em que $k^s = {\tt sc\textunderscore reduce32}(\mathit{Keccak}(a))$ e $k^v = {\tt sc\textunderscore reduce32}(\mathit{Keccak}(\mathit{Keccak}(a)))$. \cite{luigi-address}}  
%It\marginnote{src/wallet/ api/wallet.cpp {\tt create()} wallet2.cpp {\tt generate()} {\tt get\_seed()}} is currently most common for the view key $k^v$ to equal $\mathcal{H}_n(k^s)$. This means a person only needs to save their spend key $k^s$ in order to access (view and spend) all of the outputs they own (spent and unspent). The spend key is typically represented as a series of 25 words (where the 25th word is a checksum). Other, less popular methods include: generating $k^v$ and $k^s$ as separate random numbers, or generating a random 12-word mnemonic $a$, where $k^s = {\tt sc\textunderscore reduce32}(\mathit{Keccak}(a))$ and $k^v = {\tt sc\textunderscore reduce32}(\mathit{Keccak}(\mathit{Keccak}(a)))$. \cite{luigi-address}}



\section{Endereços ocultos}
\label{sec:one-time-addresses}

Para receber dinheiro, um moneriano pode distribuír o seu endereço a outros. Estes por sua vez enviam-lhe dinheiro contido nas {\em saídas} das transacções\newline(ou contido numa saída de uma transacção, no caso singular de ele só conhecer um outro moneriano. E de este só lhe enviar uma vez dinheiro numa transacção com só uma saída).
\footnote{O método descrito aqui não é enforcado pelo protocolo, só por implementações standard de carteira. Isto significa que uma carteira alternativa pode seguir o estilo de Bitcoin em que os endereços dos recipientes estão incluídos directamente nos dados de transacção. Tal carteira alternativa iria produzir saídas de transacção que não são usáveis por outras carteiras, e cada tal endereço tipo Bitcoin só poderia ser usado uma vez porque as imagens de chave têm de ser únicas.}
%The method described here is not enforced by the protocol, just by wallet implementation standards. This means an alternate wallet could follow the style of Bitcoin where recipients' addresses are included directly in transaction data. Such a non-compliant wallet would produce transaction outputs unusable by other wallets, and each Bitcoin-esque address could only be used once due to key image uniqueness.}
\\
\centerline{{\bf Um endereço nunca está na lista de blocos.}}
\newline
\newline
Em vez disso é aplicada uma troca tipo Diffie-Hellman ao endereço, cada vez que se faz uma transacção para esse endereço. O que gera assim um {\em endereço oculto} único, para cada {\em saída} de transacção.\newline
Desta forma, observadores externos, que conhecem todos os endereços de um certo moneriano, são incapazes de verificar quais são as {\em saídas} de transacções que lhe pertencem (enp).

%A recipient can spend their received outputs by signing a message with the one-time addresses, thereby proving they know the private keys and therefore own what they are spending. We will gradually flesh out this concept.
%\\

Comecemos com o caso base, uma transacção simples da alice para o bob, contendo só uma {\em saída}, com o montante de `0.123' xmr.\newline
O bob tem as chaves $(k_B^s, k_B^v)$ e $(K_B^s, K_B^v)$, e a alice sabe o seu endereço.
Esta transacção prosegue da seguinte maneira :\cite{cryptoNoteWhitePaper}:

\begin{enumerate}
	\item A alice gera um número aleatório $r \in_R \mathbb{Z}_l$, e calcula o {\em endereço oculto} :
\footnote{Em Monero cada instância de $r k^v G$ é multiplicada pelo cofactor 8, assim neste caso a alice computa $8*r K^v_B$ e o bob computa $8*k^v_B r G$. Multiplicar pelo cofactor de 8 garante que o ponto resultante está no subgrupo de $G$. Se $R$ e $K^v$ não estão dentro do mesmo subgrupo, então os logaritmos discretos de $r$ e $k^v$ não podem ser usados para fazer um segredo partilhado.} 

%In\marginnote{src/crypto/ crypto.cpp {\tt generate\_ key\_deri- vation()}} Monero, every instance (including when it's used for other parts of the transaction) of $r k^v G$ is multiplied by the cofactor 8, so in this case Alice computes $8*r K^v_B$ and Bob computes $8*k^v_B r G$. As far as we can tell this serves no purpose (but it {\em is} a rule that users must follow). Multiplying by the cofactor ensures the resulting point is in $G$'s subgroup, but if $R$ and $K^v$ don't share a subgroup to begin with, then the discrete logs $r$ and $k^v$ can't be used to make a shared secret regardless.}
\vspace{.175cm}
\begin{align*}
K^o  = \mathcal{H}_n(r K_B^v)G + K_B^s
\end{align*}
\marginnote{src/crypto/ crypto.cpp {\tt derive\_pu- blic\_key()}}

	\item A alice pôe $K^o$ como o destinatário do pagamento, adiciona o montante da {\em saída} `0.123' xmr, e o valor $r G$ aos dados da transacção, e submete isso á rede monero.

	\item O bob\marginnote{src/crypto/ crypto.cpp {\tt derive\_ subaddress\_ public\_ key()}} recebe esses dados e vê $r G$ e $K^o$.\newline Ele pode calcular : 
	\begin{align*}
	r G k_B^v &= r K_B^v.
	\end{align*}
\newline Ele depois calcula :
	\begin{align*}
	K'^s_B = K^o - \mathcal{H}_n(r K_B^v)G.
	\end{align*}
Ao ver que : 
	\begin{align*}
	K'^s_B = K_B^s.
	\end{align*}
O bob sabe que essa saída lhe pertence.\footnote{$K'^s_B $ é calculado com {\tt derive\_subaddress\_public\_key()} porque as chaves de gasto de endereços normais são guardados numa {\em tabela de chaves de gasto} com o index 0, enquanto que os sub-endereços começam com 1,2,3 ... isto irá fazer sentido brevemente : veja-se a secção \ref{sec:subaddresses}.} 
%$K'^s_B $ is computed with {\tt derive\_subaddress\_public\_key()} because normal address spend keys are stored at index 0 in the spend key lookup table, while subaddresses are at indices 1,2... This will make sense soon: see Section \ref{sec:subaddresses}.}
\newline Note-se aqui que sabendo o endereço do bob, quem possuir também a chave privada $k_B^v$, pode calcular $K'^s_B$ para cada {\em saída} de transacção na lista de blocos, e ver quais pertencem ao bob.

	\item O {\em endereço oculto} e a {\em chave oculta} são :\vspace{.175cm}
	\begin{align*}
		K^o &= \mathcal{H}_n(r K_B^v)G + K_B^s = (\mathcal{H}_n(r K_B^v) + k_B^s)G  \\ 
		k^o &= \mathcal{H}_n(r K_B^v) + k_B^s
	\end{align*}
\end{enumerate}

Para gastar o seu montante `0.123' numa nova transacção, o bob precisa de fazer uma assinatura com a sua chave privada oculta $k^o$.
%To spend his `0.123' amount output [sic] in a new transaction, all Bob needs to do is prove ownership by signing a message with the one-time key $k^o$. 
Sem esta chave a alice não têm a certeza se o bob gastou a saída que ela lhe enviou.
%As will become clear in Chapter \ref{chapter:transactions}, without $k^o$ Alice can't compute the output's key image, so she can never know for sure if Bob spends the output she sent him.
\footnote{Imagine-se que a alice produz duas transacções, cada uma delas contêm o mesmo endereço oculto $K^o$ que o bob pode gastar. Desde que $K^o$ só depende de $r$ e $K_B^v$, ela consegue de facto fazer isto se ela quisesse. O bob so pode gastar uma dessas saídas, porque cada endereço oculto só tem uma imagem de chave. A alice faz então uma transacção 1 com um montante enorme e uma transacção 2 com um montante pequeno. Se ele gasta o montante de 2, nunca poderá gastar o montante em 1. Aliás ninguêm mais pode gastar o dinheiro em 1, ou seja esse montante `arde'. As carteiras em Monero foram feitas para ignorar o montante mais pequeno nesta situação.}      

%Imagine Alice produces two transactions, each containing the same one-time output address $K^o$ that Bob can spend. Since $K^o$ only depends on $r$ and $K_B^v$, there is no reason she can't do it. Bob can only spend one of those outputs because each one-time address only has one key image, so if he isn't careful Alice might trick him. She could make transaction 1 with a lot of money for Bob, and later transaction 2 with a small amount for Bob. If he spends the money in 2, he can never spend the money in 1. In fact, no one could spend the money in 1, effectively `burning' it. Monero wallets have been designed to ignore the smaller amount in this scenario.}

O bob pode dar a sua {\em chave de ver} a qualquer entidade terceira, como por exemplo um auditor, entidade contabilística ou autoridade tributária etc...
Ou seja a alguêm que tenha direitos de ler o seu histórico de transacções. Ao fazer isto o bob está a confiar que essa chave privada não vá parar ás maos erradas. Pois a chave de ver também permite ver o montante nas {\em saídas} de transacções. (O que será explicado na secção \ref{sec:pedersen_monero}). Veja o capítulo \ref{chapter:tx-knowledge-proofs} para outros meios com os quais o bob pode fornecer informações sobre o seu histórico de transacções.


\subsection{transacções de múltiplas saídas}
\label{sec:multi_out_transactions}

A maioria das transacções irá conter mais do que uma saída, por exepmlo duas saídas em que uma delas serve para que o remetente receba de volta o seu troco.
\footnote{Actualmente, cada transacção {\em requer} sempre duas saídas, mesmo que uma delas seja com o montante de $0$ ({\tt HF\_VERSION\_MIN\_2\_OUTPUTS}). Excepto as transacções de mineiro. Isto melhora a anonimidade das transacções pois elas parecem todas ainda mais iguais. Antes de v12 a carteira do software núcleo de Monero já criava saídas com um montante nulo. A implementação núcleo envia um montante de 0 para um endereço aleatório \marginnote{src/wallet/ wallet2.cpp {\tt transfer\_ selected\_ rct()}}.}   
%Actually, as of protocol v12 two outputs are {\em required} from each (non-miner) transaction, even if it means one output has 0 amount ({\tt HF\_VERSION\_MIN\_2\_OUTPUTS}). This improves transaction indistinguishability by mixing 1-output cases with the much more common 2-output transactions. Previous to v12 the core wallet software was already creating 0-value outputs. The core implementation sends 0-amount outputs to a random address.\marginnote{src/wallet/ wallet2.cpp {\tt transfer\_ selected\_ rct()}}}
\footnote{Depois de {\em Bulletproofs} terem sido implementados no protocolo v8, cada transacção está limitada a um máximo de 16 saídas ({\tt BULLETPROOF\_MAX\_OUTPUTS}). Antes disso não havia tal limite, aparte de um limite do tamanho da transacção (em bytes)}
%After Bulletproofs were implemented in protocol v8, each transaction became limited to no more than 16 outputs ({\tt BULLETPROOF\_MAX\_OUTPUTS}). Previously there was no limit, aside from a restraint on transaction size (in bytes).}%BULLETPROOF_MAX_OUTPUTS; wallet2::transfer_selected_rct() for random address 0-amount

Os remetentes de Monero usualmente geram só um valor aleatório $r$.\newline O ponto de curva elíptica $r G$ chama-se a {\em chave pública de transacção} e é publicada juntamente com outros dados dentro da transacção.
%Monero senders usually generate only one random value $r$. The curve point $r G$ is typically known as the {\em transaction public key} and is published alongside other transaction data in the blockchain.\\

Para\marginnote{src/crypto- note\_core/ cryptonote\_ tx\_utils.cpp {\tt construct\_ tx\_with\_ tx\_key()}}[-.8cm] garantir que todos os endereços ocultos dentro de uma transacção com $p$ saídas, são únicos mesmo quando um destinatário é usado duas vezes,\newline existe um {\em index de saída}.\newline 
Cada saída de transacção possui um {\em index} $t \in \{0, ..., p-1\}$.\newline
Ao concatenar este valor, ao argumento da função hash, garante-se que o endereço oculto é único :\marginnote{src/crypto/ crypto.cpp {\tt derive\_pu- blic\_key()}}[1.2cm]
\vspace{.175cm}%construct_tx_with_tx_key() indexes 0 to p-1
	\begin{align*}
	  K_t^o &= \mathcal{H}_n(r K_t^v, t)G + K_t^s = (\mathcal{H}_n(r K_t^v, t) + k_t^s)G  \\ 
	  k_t^o &= \mathcal{H}_n(r K_t^v, t) + k_t^s
	\end{align*} 

\section{Sub-endereços}
\label{sec:subaddresses}

Monerianos podem gerar sub-endereços a partir de cada endereço príncipal.\newline Montantes enviados para um sub-endereço podem ser vistos e gastos usando a chaves de ver e de gasto do endereço principal \cite{MRL-0006-subaddresses}. Como analogia : uma conta  online no banco pode ter vários saldos correspondentes a cartões de crédito e depósitos, e o proprietário da conta é capaz de gastar e ver todos esses saldos.
% Funds sent to a subaddress can be viewed and spent using its main address’s view and spend keys. By analogy: an online bank account may have multiple balances corresponding to credit cards and deposits, yet they are all accessible and spendable from the same point of view – the account holder.\\

Sub-endereços são convenientes para receber fundos para o mesmo sítio quando um utilisador não quer que a sua actividade esteja ligada ao mesmo endereço. Como veremos, observadores externos teriam de quebrar o PLD para determinarem que um subendereço pertence a um endereço particular \cite{MRL-0006-subaddresses}.
%Subaddresses are convenient for receiving funds to the same place when a user doesn't want to link his activities together by publishing/using the same address. As we will see, most observers would have to solve the DLP to determine a given subaddress is derived from any particular address \cite{MRL-0006-subaddresses}.
\footnote{Antes do uso de sub-endereços (adicionado no update de software do protocolo v7 \cite{subaddress-pull-request}), os monerianos podiam simplesmente gerar muitos endereços normais. E para ver o saldo de cada um desses endereços era necessário fazer um scan separado á lista de blocos. Isto era muito ineficiente, com sub-endereços, mantêm-se uma tabela hash das chaves de gasto. Assim um scan da lista de blocos leva o mesmo tempo para 1 sub-endereço, ou 10,000 sub-endereços.} 
%Prior to subaddresses (added in the software update corresponding with protocol v7 \cite{subaddress-pull-request}), Monero users could simply generate many normal addresses. To view each address's balance, you needed to do a separate scan of the blockchain record. This was very inefficient. With subaddresses, users maintain a look-up table of (hashed) spend keys, so one scan of the blockchain takes the same amount of time for 1 subaddress, or 10,000 subaddresses.}
Sub-endereços também são úteis para diferenciar entre saídas recebidas. Por exemplo, se a alice quer comprar uma maçã do bob numa terça-feira, o bob podia escrever um recibo descrevendo a compra, e fazendo um subendereço para esse recibo. A alice depois envia o dinheiro para esse sub-endereço, desta forma o bob consegue associar o dinheiro que ele recebeu com a maça que ele vendeu. Uma outra forma de explorar entre saídas recebidas irá ser explorada na próxima secção. 
%They are also useful for differentiating between received outputs. For example, if Alice wants to buy an apple from Bob on a Tuesday, Bob could write a receipt describing the purchase and make a subaddress for that receipt, then ask Alice to use that subaddress when she sends him the money. This way Bob can associate the money he receives with the apple he sold. We explore another way to distinguish between received outputs in the next section.

\footnote{Antes de sub-endereços terem sido implementados em abril de 2018, os programadores de Monero suportavam um método chamado ID's de pagamento, o que permitia os recipientes de diferenciar entre as várias saídas recebidas. ID's de pagamento eram strings incluídas nos dados do campo {\em extra} de uma transacção. Os recipientes podiam pedir aos remetentes para incluir um ID de pagamento nos dados de transacção. ID's de transacção também podiam ser incluídos em endereços integrados. ID's de pagamento e endereços integrados não afectam a verificação da transacção. Portanto isso ainda pode ser implementado por qualquer pessoa, mas desde que isso está presentemente descontinuado nas carteiras mais populares escolheu-se não esplicar isso aqui.} 
%Before subaddresses were implemented in April 2018, the Monero developers supported a method called payment IDs, which allowed receivers to differentiate between owned outputs. Payment IDs were text strings included in a transaction's `extra', miscellaneous, data. Recipients could ask senders to include the payment IDs when writing a transaction. Payment IDs could also be encoded in conjunction with so-called integrated addresses. Payment IDs and integrated addresses do not affect transaction verification, so they can still be implemented by anyone, but since they are currently deprecated in the most popular wallets we have chosen not to explain them here.}

\footnote{Note-se que o index $i$ podia também simplesmente ser uma hash gerada de um password x tal que : $\mathcal{H}_n(x)$. Isto iria permitir que o proprietário de um endereço principal possa ver todos os seus montantes dos sub-endereços, mas só os consegue gastar ao escrever o password. Actualmente não existem carteiras que protejam os sub-endereços com passwords e também não existem planos para implementar tais carteiras.}
%Note that the index $i$ could just as easily be some password-generated hash $\mathcal{H}_n(x)$. This would allow an address owner to view his subaddress funds from his main address view key, but only be able to spend those funds by inputting a password. There are currently no wallet implementations password protecting subaddresses, nor any known plans to develop a wallet with that feature.}
O bob gera o seu $i${\em -ésimo} sub-endereço $(i = 1, 2, ...)$ do seu endereço como um par de chaves públicas $(K^{s,i}, K^{v,i})$ :
%Bob\marginnote{src/device/ device\_de- fault.cpp {\tt get\_sub- address\_ secret\_ key()}} generates his $i^\nth$ subaddress $(i = 1, 2, ...)$ from his address as a pair of public keys $(K^{v,i}, K^{s,i})$:
\footnote{Acontece que a hash do sub-endereço está separada por domínios, portanto é realmente $\mathcal{H}_n(T_{sub},k^v,i)$ em que $T_{sub} = $``SubAddr". Omite-se $T_{sub}$ ao longo deste documento para brevidade.}
%It turns out the subaddress hash is domain separated, so it's really $\mathcal{H}_n(T_{sub},k^v,i)$ where $T_{sub} = $``SubAddr". We omit $T_{sub}$ throughout this document for brevity.}\vspace{.175cm}
\begin{align*}
    K^{s,i} &= K^s + \mathcal{H}_n(k^v, i) G\\
    K^{v,i} &= k^v K^{s,i}
\end{align*}
\quad Assim,
\begin{alignat*}{2}
    K^{s,i} &= &&(k^s + \mathcal{H}_n(k^v, i))G\\    
    K^{v,i} &= k^v&&(k^s + \mathcal{H}_n(k^v, i))G
\end{alignat*}
    

\subsection{Enviar para um sub-endereço}
    
Seja que a alice envia ao bob o montante 0.123 outra vez, desta vez para o seu sub-endereço $(K_B^{v,1}, K_B^{s,1})$. 
%Let's say Alice is going to send Bob `0' amount again, this time via his subaddress $(K_B^{v,1}, K_B^{s,1})$.
\begin{enumerate}
	\item a alice gera um número aleatório $r \in_R \mathbb{Z}_l$, e calcula o endereço oculto :
    \vspace{.175cm}
	\[ K^o  = \mathcal{H}_n(r K_B^{v,1},0)G + K_B^{s,1} \]
%Alice generates a random number $r \in_R \mathbb{Z}_l$, and calculates the one-time address\vspace{.175cm}
%	\[ K^o  = \mathcal{H}_n(r K_B^{v,1},0)G + K_B^{s,1} \]

	\item A alice põem $K^o$ como o destinatário do pagamento, adiciona o montante `0.123' e o valor $r K_B^{s,1}$ aos dados de transacção, e submete isso á rede. 
%Alice sets $K^o$ as the addressee of the payment, adds the output amount `0' and the value $r K_B^{s,1}$ to the transaction data, and submits it to the network.
	
	\item O bob recebe os dados e vê os valores :
    \begin{align*}
    r K_B^{s,1},\ K^o.
	\end{align*}
    \item Ele calcula então :
    \begin{align*}
    k_B^v r K_B^{s,1} &= r K_B^{v,1}\\
    K'^{s}_B &= K^o - \mathcal{H}_n(r K_B^{v,1},0)G	
    \end{align*}
    \item Quando ele vê que :
    \begin{align*}
    K'^{s}_B = K^{s,1}_B ,	
    \end{align*}
    então o bob sabe que a transacção é dirigida para ele.%Bob\marginnote{src/crypto/ crypto.cpp {\tt derive\_ subaddress\_ public\_ key()}} receives the data and sees the values $r K_B^{s,1}$ and $K^o$. He can calculate $k_B^v r K_B^{s,1} = r K_B^{v,1}$. He can then calculate $K'^{s}_B = K^o - \mathcal{H}_n(r K_B^{v,1},0)G$. When he sees that $K'^{s}_B = K^{s,1}_B$, he knows the transaction is addressed to him.
%\footnote{Um ataquante sofisticado pode ser capaz de ligar sub-endereços \cite{janus-attack} ({\em ataque de Janus}). An advanced attacker may be able to link subaddresses \cite{janus-attack} (a.k.a. the Janus attack). With subaddresses (one of which can be a normal address) $K_B^1$ $\&$ $K_B^2$ the attacker thinks may be related, he makes a transaction output with $K^o = \mathcal{H}_n(r K_B^{v,2},0)G + K_B^{s,1}$ and includes transaction public key $r K_B^{s,2}$. Bob calculates $r K_B^{v,2}$ to find $K'^{s,1}_B$ but has no way of knowing it was his {\em other} (sub)address's key used! If he tells the attacker that he received funds to $K_B^1$, the attacker will know $K_B^2$ is a related subaddress (or normal address). Since subaddresses are outside the protocol's scope, mitigations are up to wallet implementers. No known wallets have done so, and any mitigation would only work for compliant wallets. Potential mitigations include: not informing attackers of received funds, including encrypted transaction private key $r$ in transaction data, a Schnorr signature on the shared secret using $K^{s,1}$ as the base point instead of $G$, and including $rG$ in transaction data and verifying the shared secret with $rK^{s,1} \stackrel{?}{=} (k^s + \mathcal{H}_n(k^v, 1))*rG$ (requires the private spend key). Outputs received by a normal address should also be verified. See \cite{janus-mitigation-issue-62} for a discussion of the last mitigation listed.}
	
    O bob só precisa a sua chave privada de ver $k_B^v$ e o sub-endereço público de gasto $K^{s,1}_B$ para encontrar saídas de transacção enviadas ao seu sub-endereço.   
%	Bob only needs his private view key $k_B^v$ and subaddress public spend key $K^{s,1}_B$ to find transaction outputs sent to his subaddress.
	
	\item As chaves ocultas para a saída são :
%The one-time keys for the output are:\vspace{.175cm}
	\begin{align*}
		K^o &= \mathcal{H}_n(r K_B^{v,1},0)G + K_B^{s,1} = (\mathcal{H}_n(r K_B^{v,1},0) + k_B^{s,1})G  \\ 
		k^o &= \mathcal{H}_n(r K_B^{v,1},0) + k_B^{s,1}
	\end{align*}
\end{enumerate}

A chave pública de transacção é particular ao bob e em vez de ser $r G$ é :
\begin{align*}
r K_B^{s,1} .
\end{align*}
Por outras palavras seja que a alice quer enviar numa só transacção dinheiro para $p$ saídas, se pelo menos uma das saídas é para um sub-endereço, então é necessário fazer uma chave pública de transacção para cada saída.\newline Por exemplo a alice envia xmr para um sub-endereço $(K_B^{v,1}, K_B^{s,1})$ de bob e para um endereço $(K_C^v, K_C^s)$ da carol, então ela põe \{$r_1 K_B^{s,1},r_2 G$\} nos dados de transacção.
%Now,\marginnote{src/crypto- note\_core/ cryptonote\_ tx\_utils.cpp {\tt construct\_ tx\_with\_ tx\_key()}} Alice's transaction public key is particular to Bob ($r K_B^{s,1}$ instead of $r G$). If she creates a $p$-output transaction with at least one output intended for a subaddress, Alice needs to make a unique transaction public key for each output $t \in \{0,...,p-1\}$. In other words, if Alice is sending to Bob's subaddress $(K_B^{v,1}, K_B^{s,1})$ and Carol's address $(K_C^v, K_C^s)$, she will put two transaction public keys \{$r_1 K_B^{s,1},r_2 G$\} in the transaction data.
\footnote{Em Monero\marginnote{src/crypto- note\_basic/ cryptonote\_ basic\_impl.cpp {\tt get\_account\_ address\_as\_ str()}} os sub-endereços levam o prefixo de 8 enquanto que os endereços levam o prefixo de 4. Isto ajuda aos remetentes escolherem o processo correcto ao construir transacções.}
%In\marginnote{src/crypto- note\_basic/ cryptonote\_ basic\_impl.cpp {\tt get\_account\_ address\_as\_ str()}} Monero subaddresses are prefixed with an ‘8’, separating them from addresses, which are prefixed with ‘4’. This helps senders choose the correct procedure when constructing transactions.}
\footnote{Existem aspectos intrincados, com o uso de chaves públicas adicionais de transacção, ao redor de saídas de troco em que o autor da transacção sabe a chave de ver do destinatário (desde que é ele próprio; também o caso para saídas de troco em que o montante é nulo).\newline Sempre que existem pelo menos duas saídas que não são de troco, e pelo menos um dos destinatários é um sub-endereço, procede-se da froma normal como descrito acima (actualmente um erro na implementação núcleo adiciona uma chave pública de transacção, aos dados da transacção, o que é inútil).\newline
Se somente a saída de troco é para um sub-endereço, ou só existe uma saída na transacção (sem troco) e esta é para um sub-endereço, então só se gera uma chave pública de transacção. No caso anterior, a chave pública de transacção é $rG$ e o sub-endereço oculto de troco é : $K^o = \mathcal{H}_n(k^v_c r G,t)G + K_c^{s,1}$ . O que usa a chave de ver normal e a chave de gasto do sub-endereço. No caso posterior a chave pública de transacção $r K^{v,1}$ baseia-se da chave de ver do sub-endereço, e o endereço oculto de troco é : $K^o = \mathcal{H}_n(k^v_c*r K^{v,1},t)G + K_c^s$. Estes detalhes, ajudam a misturar uma porção de transacções de sub-endereços entre as transacções mais comuns. Transacções com duas saídas de endereços normais perfazem pelo menos 95\% do volume de transacções.} 
%These details help mingle a portion of subaddress transactions amongst the more common normal address transactions, and 2-output transactions which compose around 95\% of transaction volume as of this writing.}
%In the latter case the transaction public key $r K^{v,1}$ is based off the subaddress's view key, and the change's one time key is $K^o = \mathcal{H}_n(k^v_c*r K^{v,1},t)G + K_c^s$.
%If either just the change output is to a subaddress, or there is just one non-change output and it's to a subaddress, then only one transaction public key is created. In the former case, the transaction public key is $rG$ and the change's one-time key is (subaddress index 1, using $c$ subscript to denote the change's keys) $K^o = \mathcal{H}_n(k^v_c r G,t)G + K_c^{s,1}$ using the normal view key and subaddress's spend key.

%There is some intricacy to when additional transaction public keys are used (see the code path {\tt transfer\_selected\_rct()} $\rightarrow$ {\tt construct\_tx\_and\_get\_tx\_key()} $\rightarrow$ {\tt construct\_tx\_with\_tx\_key()} $\rightarrow$ {\tt generate\_output\_ephemeral\_keys()} and {\tt classify\_addresses()}) surrounding change outputs where the transaction author knows the recipient's view key (since it's himself; also the case for dummy change outputs, which are created when a 0-amount output is necessary, since authors generate those addresses). Whenever there are at least two non-change outputs, and at least one of their recipients is a subaddress, proceed in the normal way explained above (a current bug in the core implementation adds an extra transaction public key to transaction data even beyond the additional keys, which is not used for anything). If either just the change output is to a subaddress, or there is just one non-change output and it's to a subaddress, then only one transaction public key is created. In the former case, the transaction public key is $rG$ and the change's one-time key is (subaddress index 1, using $c$ subscript to denote the change's keys) $K^o = \mathcal{H}_n(k^v_c r G,t)G + K_c^{s,1}$ using the normal view key and subaddress's spend key. In the latter case the transaction public key $r K^{v,1}$ is based off the subaddress's view key, and the change's one time key is $K^o = \mathcal{H}_n(k^v_c*r K^{v,1},t)G + K_c^s$. These details help mingle a portion of subaddress transactions amongst the more common normal address transactions, and 2-output transactions which compose around 95\% of transaction volume as of this writing.}



\section{Endereços integrados}
\label{sec:integrated-addresses}

Para diferenciar entre saídas recebidas, um destinatário pode requisitar dos remetentes que incluam um {\em ID de pagamento} nos dados de transacção.
%To differentiate between the outputs they receive, a recipient can request senders include a {\em payment ID} in transaction data.
\footnote{Actualmente, as implementações de Monero só permitem o suporte de um ID de pagamento por transacção, independentemente do número de saídas.} 
%Currently, Monero implementations only support one payment ID per transaction regardless of how many outputs it has.} 
Por exemplo se a alice quer comprar uma maçã ao bob numa terça-feira, o bob pode escrever um recibo que descreve a compra, e pedir a alice que inclua tal número de ID do recibo nos dados de transacção, quando ela lhe envia o dinheiro. Desta forma o bob consegue associar o dinheiro que recebe com a maça que vendeu.
%For example, if Alice wants to buy an apple from Bob on a Tuesday, Bob could write a receipt describing the purchase and ask Alice to include the receipt's ID number when she sends him the money. This way Bob can associate the money he receives with the apple he sold.

A um dado ponto os remetentes podiam comunicar os ID's de pagamento em texto claro e não encriptado. Mas incluir os ID's de pagamento de forma manual é inconveniente, e texto claro é um perigo para a privacidade dos destinatários, que desta forma podem expor as suas actividades.
%At one point senders could communicate payment IDs in clear text, but manually including the IDs in transactions is inconvenient, and cleartext is a privacy hazard for recipients, who might inadvertently expose their activities.
\footnote{Estes ID's de pagamento de longo formato (256 bits) foram descontinuados na v0.15 do software núcleo (o que coíncide com o protocolo v12 em Novembro de 2019). Enquanto que outras carteiras talvez ainda suportem a inclusão destes ID's nos dados de transacção, a carteira núcleo irá ignorá-los.}
%These long-form (256 bit) cleartext payment IDs were deprecated in v0.15 of the core software implementation (coincident with protocol v12 in November 2019). While other wallets may still support them and allow their inclusion in transaction data, the core wallet will ignore them.} 
Em\marginnote{src/crypto- note\_basic/ cryptonote\_ basic\_ impl.cpp {\tt get\_ account\_ integrated\_ address\_as\_ str()}} Monero, os destinatários podem integrar os ID's de pagamento nos seus endereços, e oferecer estes {\em endereços integrados}, que contêm : ($K^s$, $K^v$, ID), aos remetentes. Estes ID's de pagamento podem ser integrados em qualquer tipo de endereço, endereços normais, sub-endereços e endereços de multi-assinatura. 
%In\marginnote{src/crypto- note\_basic/ cryptonote\_ basic\_ impl.cpp {\tt get\_ account\_ integrated\_ address\_as\_ str()}} Monero, recipients can integrate payment IDs into their addresses, and provide those {\em integrated addresses}, containing ($K^v$, $K^s$, payment ID), to senders. Payment IDs can technically be integrated into any kind of address, including normal addresses, subaddresses, and multisignature addresses.
\footnote{Tanto quanto o autor saiba, endereços integrados só foram implementados para endereços normais.}
%Within the authors' knowledge, integrated addresses have only ever been implemented for normal addresses.}

Remetentes que enviam saídas para endereços integrados codificam ID's de pagamento usando o segredo partilhado $r K_t^v$ e uma operação XOR (secção \ref{sec:XOR_section}), os recipientes por sua vez descodificam isso com a chave pública de transacção e outra operação XOR \cite{integrated-addresses}. Isto permite a remetentes provar que criaram certas saídas de transacção (e.g. para auditorias, reembolsos, etc...).   
%Senders addressing outputs to integrated addresses can encode payment IDs using the shared secret $r K_t^v$ and an XOR operation (recall Section \ref{sec:XOR_section}), which recipients can then decode with the appropriate transaction public key and another XOR operation \cite{integrated-addresses}. Encoding payment IDs in this way also allows senders to prove they made particular transaction outputs (e.g. for audits, refunds, etc.).
\footnote{Desde que um observador consegue reconhecer a diferença entre transacções com e sem ID's de pagamento, pensa-se que usar ID's torna o histórico de transacções em Monero menos uniforme. Por causa disto, desde o protocolo v10 a implementação núcleo adiciona um ID de pagamento codificado desvio, para todas as transacções com duas saídas. Estas descodificadas revelam só zeros (isto não é requisito do protocolo, somente melhor prática).}
%Since an observer can recognize the difference between transactions with and without payment IDs, using them is thought to make the Monero transaction history less uniform. Because of this, since protocol v10 the core implementation adds a dummy\marginnote{src/crypto- note\_core/ cryptonote\_ tx\_utils.cpp {\tt construct\_ tx\_with\_ tx\_key()}} encrypted payment ID to all 2-output transactions. Decoding one will reveal all 0s (this is not required by the protocol, just best practice).}


\subsubsection*{Codificar}

O remetente codifica o ID de pagamento para inclusão nos dados de transacção
\marginnote{src/device/ device\_de- fault.cpp {\tt encrypt\_ payment\_ id()}}
%The sender encodes the payment ID for inclusion in transaction data
\footnote{Em Monero os ID's de pagamento usualmente são 64 bits.} 
%In Monero payment IDs for integrated addresses are conventionally 64 bits long.}
\vspace{.175cm}
\begin{align*}
         k_{\textrm{máscara}} &= \mathcal{H}_n(r K_t^v,\textrm{pid\_tag}) \\
      k_{\textrm{ID}} &= k_{\textrm{máscara}} \rightarrow \textrm{reduzido a 64 bits}\\
  \textrm{ID codificado} &= \textrm{XOR}(k_{\textrm{ID}}, \textrm{ID})
\end{align*}

Incliu-se o pid\_tag para garantir que $k_{\textrm{mask}}$ é diferente do componente $\mathcal{H}_n(r K_t^v, t)$ em endereços ocultos.
%We include pid\_tag to ensure $k_{\textrm{mask}}$ is different from the component $\mathcal{H}_n(r K_t^v, t)$ in one-time output addresses.
\footnote{Em Monero, pid\_tag = {\tt cauda\_do\_ID\_codificado} = 141.}
%In Monero, pid\_tag = {\tt ENCRYPTED\_PAYMENT\_ID\_TAIL} = 141. 

%In, for example, multi-input transactions we compute $\mathcal{H}_n(r K_t^v, t) \pmod l$ to ensure we are using a scalar less than the EC subgroup order, but since $l$ is 253 bits and payment IDs are only 64 bits, taking the modulus for encoding payment IDs would be meaningless, so we don't.}
%??? o queeee?

\subsubsection*{Descodificar}

O destinatário usa a chave pública de transacção $r G$, e a sua chave de ver para encontrar um ID de pagamento.
%Whichever\marginnote{src/device/ device.hpp {\tt decrypt\_ payment\_ id()}} recipient the payment ID was created for can find it using his view key and the transaction public key $r G$:
\footnote{Os dados de transacção não explicitam para qual saída pertence um ID de pagamento. Os destinatários têm de identificar os seus próprios ID's}
%Transaction data does not indicate which output a payment ID `belongs' to. Recipients have to identify their own payment IDs.}
\vspace{.175cm}
\begin{align*}
         k_{\textrm{máscara}} &= \mathcal{H}_n(k_t^v r G,\textrm{pid\_tag}) \\
      k_{\textrm{ID}} &= k_{\textrm{máscara}} \rightarrow \textrm{reduzido a 64 bits}\\
          \textrm{ID} &= \textrm{XOR}(k_{\textrm{ID}}, \textrm{ID codificado})
\end{align*}


%Similarly, senders can decode payment IDs they had previously encoded by recalculating the shared secret.%remove this line?yes
\iffalse

\section{Endereços de multi-assinatura}
\label{sec:multisignature-addresses}

Ás vezes é útil partilhar a posse de fundos entre várias pessoas/endereços. O capítulo \ref{chapter:multisignatures} é dedicado a este tópico.
%Sometimes it is useful to share ownership of funds between different people/addresses. We dedicate Chapter \ref{chapter:multisignatures} to elaborating this topic.
\fi
