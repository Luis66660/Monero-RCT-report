\begin{appendices}

\renewcommand{\theFancyVerbLine}{%
	\textcolor{red}{\small
		\arabic{FancyVerbLine}}}

\chapter{Estrutura de transacção {\tt RCTTypeBulletproof2}}
\label{appendix:RCTTypeBulletproof2}
%\addcontentsline{toc}{chapter}{Estrutura de transacção {\tt RCTTypeBulletproof2}}

Apresenta-se neste apêndice uma transacção de Monero real do tipo {\tt RCTTypeBulletproof2}, juntamente com as esplicações dos seus campos relevantes.
%We present in this appendix a dump from a real Monero transacção of type {\tt RCTTypeBulletproof2}, together with explanatory notes for relevant fields.

Esta transacção foi obtida do explorador de blocos \url{https://xmrchain.net}. Também se pode obter esta transacção ao executar o commando {\tt print\_tx <TransactionID> +hex +json} no {\em daemon} {\tt monerod} (sem o parámetro {\em detached}). Em que {\tt <TransactionID>} é uma hash da transacção (secção \ref{subsec:transacção-id}). A primeira linha mostra o commando que se executou. 
%The dump was obtained from the block explorer \url{https://xmrchain.net}. It can also be acquired by executing command {\tt print\_tx <TransactionID> +hex +json} in the {\tt monerod} daemon run in non-detached mode. {\tt <TransactionID>} is a hash of the transacção (Section \ref{subsec:transacção-id}). The first linha printed shows the actual command to run.% see chapter 7 blockchain on TransactionID
Para replicar estes resultados, o leitor faz o seguinte :
%To replicate our results, readers can do the following:%\footnote{Blockchain data can also be found with a web block explorer, such as \url{https://moneroblocks.info/} or \url{https://xmrchain.net}.}
\begin{enumerate}
    \item Precisa-se a {\em cli} de Monero, que se encontra em \url{https://web.getmonero.org/downloads/}. Saque somente a {\em cli}, para o seu sistema operativo, mova o ficheiro para um directório qualquer, e extrai-a o arquivo. 
%We need the Monero command linha tool (CLI), which can be found at \url{https://web.getmonero.org/downloads/} (among other places). Get the `Command Line Tools Only' for your operating system, move the file to a useful location, and unzip it.
    \item Abra um terminal/linha de comandos e navegue para o directório criado pela extração.
%Open the terminal/command linha and navigate into the folder created by unzipping.
    \item Corra {\tt monerod} com {\tt ./monerod}. Agora a lista de blocos irá ser saquada da rede para a sua máquina local. Infelizmente não existe outra forma de imprimir transacções localmente sem ter posse da lista inteira de blocos.    
%Run {\tt monerod} with {\tt ./monerod}. Now the Monero blockchain will download. Unfortunately, there is currently no easy way to print transactions on your own system (e.g. without using a block explorer) without downloading the blockchain.
    \item Depois do processo de sincronização estar feito, é possível executar comandos como {\tt print\_tx}. Use {\tt help} para conhecer outros comandos.
%After the syncing process is done, commands like {\tt print\_tx} will work. Use {\tt help} to learn other commands.
\end{enumerate}

%For editorial reasons we have shortened long hexadecimal chains, presenting only the beginning and end as in {\tt 0200010c7f[...]409}.

O componente {\tt rctsig\_prunable}, é podável da lista de blocos. Isto é, uma vez que há consenso sobre um bloco e o comprimento actual da lista de blocos proíbe a possibilidade de um ataque de duplo gasto, este campo inteiro pode ser podado e substituido pela hash da raíz da arvore de {\em merkle}. 

%Component {\tt rctsig\_prunable}, as indicated by its name, is {\sl prunable} from the blockchain. That is, once a block has been consensuated and the current chain length rules out all possibilities of double-spending attacks, this whole field can be pruned and replaced with its hash for the Merkle tree.

Imagens de chave são guardadas separadamente, na área não podável das transacções. Estas imagens de chave são cruciais para detectar ataques de duplo gasto e não podem ser podadas. 
%Key images are stored separately, in the non-prunable area of transactions. These are essential for detecting double-spend attacks and can’t be pruned away.
\\
Esta transacção exemplo têm duas entradas e duas saídas, e foi adicionada á lista de blocos com o carimbo no tempo de 2020-03-02 19:01:10 UTC . 
%Our sample transacção has 2 inputs and 2 outputs, and was added to the blockchain at timestamp 2020-03-02 19:01:10 UTC (as reported by its block's miner).

\begin{Verbatim}[commandchars=\\\{\}, numbers=left]
print_tx 84799c2fc4c18188102041a74cef79486181df96478b717e8703512c7f7f3349
Found in blockchain at height 2045821
\{
  "version": 2, 
  "unlock_time": 0, 
  "vin": [ \{
      "key": \{
        "amount": 0, 
        "key_offsets": [ 14401866, 142824, 615514, 18703, 5949, 22840, 5572, 16439,
        983, 4050, 320
        ], 
        "k_image": "c439b9f0da76ca0bb17920ca1f1f3f1d216090751752b091bef9006918cb3db4"
      \}
    \}, \{
      "key": \{
        "amount": 0, 
        "key_offsets": [ 14515357, 640505, 8794, 1246, 20300, 18577, 17108, 9824, 581,
        637, 1023
        ], 
        "k_image": "03750c4b23e5be486e62608443151fa63992236910c41fa0c4a0a938bc6f5a37"
      \}
    \}
  ], 
  "vout": [ \{
      "amount": 0, 
      "target": \{
        "key": "d890ba9ebfa1b44d0bd945126ad29a29d8975e7247189e5076c19fa7e3a8cb00"
      \}
    \}, \{
      "amount": 0, 
      "target": \{
        "key": "dbec330f8a67124860a9bfb86b66db18854986bd540e710365ad6079c8a1c7b0"
      \}
    \}
  ], 
  "extra": [ 1, 3, 39, 58, 185, 169, 82, 229, 226, 22, 101, 230, 254, 20, 143,
  37, 139, 28, 114, 77, 160, 229, 250, 107, 73, 105, 64, 208, 154, 182, 158, 200,
  73, 2, 9, 1, 12, 76, 161, 40, 250, 50, 135, 231
  ], 
  "rct_signatures": \{
    "type": 4, 
    "txnFee": 32460000, 
    "ecdhInfo": [ \{
        "amount": "171f967524e29632"
      \}, \{
        "amount": "5c2a1a9f54ccf40b"
      \}], 
    "outPk": [ "fed8aded6914f789b63c37f9d2eb5ee77149e1aa4700a482aea53f82177b3b41",
    "670e086e40511a279e0e4be89c9417b4767251c5a68b4fc3deb80fdae7269c17"]
  \}, 
  "rctsig_prunable": \{
    "nbp": 1, 
    "bp": [ \{
        "A": "98e5f23484e97bb5b2d453505db79caadf20dc2b69dd3f2b3dbf2a53ca280216", 
        "S": "b791d4bc6a4d71de5a79673ed4a5487a184122321ede0b7341bc3fdc0915a796", 
        "T1": "5d58cfa9b69ecdb2375647729e34e24ce5eb996b5275aa93f9871259f3a1aecd", 
        "T2": "1101994fea209b71a2aa25586e429c4c0f440067e2b197469aa1a9a1512f84b7", 
        "taux": "b0ad39da006404ccacee7f6d4658cf17e0f42419c284bdca03c0250303706c03", 
        "mu": "cacd7ca5404afa28e7c39918d9f80b7fe5e572a92a10696186d029b4923fa200", 
        "L": [ "d06404fc35a60c6c47a04e2e43435cb030267134847f7a49831a61f82307fc32",
        "c9a5932468839ee0cda1aa2815f156746d4dce79dab3013f4c9946fce6b69eff",
        "efdae043dcedb79512581480d80871c51e063fe9b7a5451829f7a7824bcc5a0b",
        "56fd2e74ac6e1766cfd56c8303a90c68165a6b0855fae1d5b51a2e035f333a1d",
        "81736ed768f57e7f8d440b4b18228d348dce1eca68f969e75fab458f44174c99",
        "695711950e076f54cf24ad4408d309c1873d0f4bf40c449ef28d577ba74dd86d",
        "4dc3147619a6c9401fec004652df290800069b776fe31b3c5cf98f64eb13ef2c"
        ], 
        "R": [ "7650b8da45c705496c26136b4c1104a8da601ea761df8bba07f1249495d8f1ce",
        "e87789fbe99a44554871fcf811723ee350cba40276ad5f1696a62d91a4363fa6",
        "ebf663fe9bb580f0154d52ce2a6dae544e7f6fb2d3808531b0b0749f5152ddbf",
        "5a4152682a1e812b196a265a6ba02e3647a6bd456b7987adff288c5b0b556ec6",
        "dc0dcb2e696e11e4b62c20b6bfcb6182290748c5de254d64bf7f9e3c38fb46c9",
        "101e2271ced03b229b88228d74b36088b40c88f26db8b1f9935b85fb3ab96043",
        "b14aae1d35c9b176ac526c23f31b044559da75cf95bc640d1005bfcc6367040b"
        ], 
        "a": "4809857de0bd6becdb64b85e9dfbf6085743a8496006b72ceb81e01080965003", 
        "b": "791d8dc3a4ddde5ba2416546127eb194918839ced3dea7399f9c36a17f36150e", 
        "t": "aace86a7a1cbdec3691859fa07fdc83eed9ca84b8a064ca3f0149e7d6b721c03"
      \}
    ], 
    "MGs": [ \{
        "ss": [[ "d7a9b87cfa74ad5322eedd1bff4c4dca08bcff6f8578a29a8bc4ad6789dee106",
        "f08e5dfade29d2e60e981cb561d749ea96ddc7e6855f76f9b807842d1a17fe00"],
        ["de0a86d12be2426f605a5183446e3323275fe744f52fb439041ad2d59136ea0b",
        "0028f97976630406e12c54094cbbe23a23fe5098f43bcae37339bfc0c4c74c07"],
        ["f6eef1f99e605372cb1ec2b3dd4c6e56a550fec071c8b1d830b9fda34de5cc05",
        "cd98fc987374a0ac993edf4c9af0a6f2d5b054f2af601b612ea118f405303306"],
        ["5a8437575dae7e2183a1c620efbce655f3d6dc31e64c96276f04976243461e08",
        "5090103f7f73a33024fbda999cd841b99b87c45fa32c4097cdc222fa3d7e9502"],
        ["88d34246afbccbd24d2af2ba29d835813634e619912ea4ca194a32281ac14e0e",
        "eacdf59478f132dd8dbb9580546f96de194092558ffceeff410ee9eb30ce570e"],
        ["571dab8557921bbae30bda9b7e613c8a0cff378d1ec6413f59e4972f30f2470d",
        "5ca78da9a129619299304d9b03186233370023debfdaddcd49c1a338c1f0c50d"],
        ["ac8dbe6bb28839cf98f02908bd1451742a10c713fdd51319f2d42a58bf1d7507",
        "7347bf16cba5ee6a6f2d4f6a59d1ed0c1a43060c3a235531e7f1a75cd8c8530d"],
        ["b8876bd3a5766150f0fbc675ba9c774c2851c04afc4de0b17d3ac4b6de617402",
        "e39f1d2452d76521cbf02b85a6b626eeb5994f6f28ce5cf81adc0ff2b8adb907"],
        ["1309f8ead30b7be8d0c5932743b343ef6c0001cef3a4101eae98ffde53f46300",
        "370693fa86838984e9a7232bca42fd3d6c0c2119d44471d61eee5233ba53c20f"],
        ["80bc2da5fc5951f2c7406fce37a7aa72ffef9cfa21595b1b68dfab4b7b9f9f0c",
        "c37137898234f00bce00746b131790f3223f97960eefe67231eb001092f5510c"],
        ["01c89e07571fd365cac6744b34f1b44e06c1c31cbf3ee4156d08309345fdb20e",
        "a35c8786695a86c0a4e677b102197a11dadc7171dd8c2e1de90d828f050ec00f"]], 
        "cc": "0d8b70c600c67714f3e9a0480f1ffc7477023c793752c1152d5df0813f75ff0f"
      \}, \{
        "ss": [[ "4536e585af58688b69d932ef3436947a13d2908755d1c644ca9d6a978f0f0206",
        "9aab6509f4650482529219a805ee09cd96bb439ee1766ced5d3877bf1518370b"],
        ["5849d6bf0f850fcee7acbef74bd7f02f77ecfaaa16a872f52479ebd27339760f",
        "96a9ec61486b04201313ac8687eaf281af59af9fd10cf450cb26e9dc8f1ce804"],
        ["7fe5dcc4d3eff02fca4fb4fa0a7299d212cd8cd43ec922d536f21f92c8f93f00",
        "d306a62831b49700ae9daad44fcd00c541b959da32c4049d5bdd49be28d96701"],
        ["2edb125a5670d30f6820c01b04b93dd8ff11f4d82d78e2379fe29d7a68d9c103",
        "753ac25628c0dada7779c6f3f13980dfc5b7518fb5855fd0e7274e3075a3410c"],
        ["264de632d9cb867e052f95007dfdf5a199975136c907f1d6ad73061938f49c01",
        "dd7eb6028d0695411f647058f75c42c67660f10e265c83d024c4199bed073d01"],
        ["b2ac07539336954f2e9b9cba298d4e1faa98e13e7039f7ae4234ac801641340f",
        "69e130422516b82b456927b64fe02732a3f12b5ee00c7786fe2a381325bf3004"],
        ["49ea699ca8cf2656d69020492cdfa69815fb69145e8f922bb32e358c23cebb0f",
        "c5706f903c04c7bed9c74844f8e24521b01bc07b8dbf597621cceeeb3afc1d0c"],
        ["a1faf85aa942ba30b9f0511141fcab3218c00953d046680d36e09c35c04be905",
        "7b6b1b6fb23e0ee5ea43c2498ea60f4fcf62f70c7e0e905eb4d9afa1d0a18800"],
        ["785d0993a70f1c2f0ac33c1f7632d64e34dd730d1d8a2fb0606f5770ed633506",
        "e12777c49ffc3f6c35d27a9ccb3d9b8fed7f0864a880f7bae7399e334207280e"],
        ["ab31972bf1d2f904d6b0bf18f4664fa2b16a1fb2644cd4e6278b63ade87b6d09",
        "1efb04fe9a75c01a0fe291d0ae00c716e18c64199c1716a086dd6e32f63e0a07"],
        ["a6f4e21a27bf8d28fc81c873f63f8d78e017666adbf038da0b83c2ad04ef6805",
        "c02103455f93c2d7ec4b7152db7de00d1c9e806b1945426b6773026b4a85dd03"]], 
        "cc": "d5ac037bb78db41cf924af713b7379c39a4e13901d3eac017238550a1a3b910a"
      \}],
    "pseudoOuts": [ "b313c1ae9ca06213684fbdefa9412f4966ad192bc0b2f74ed1731381adb7ab58",
    "7148e7ef5cfd156c62a6e285e5712f8ef123575499ff9a11f838289870522423"]
  \}
\}
\end{Verbatim}



\section*{Componentes de transacção}
	
\begin{itemize}
    \item (linha 2) - O comando {\tt print\_tx} iria reportar o bloco em que foi encontrada a transacção, o que é replicado aqui com o propósito de demonstração.  
%The command {\tt print\_tx} would report the block where it found the transacção, which we replicate here for demonstration purposes.
	\item {\tt version} (linha 4) - O formato/versão da transacção; `2' corresponde a RingCT.
%Transacção format/era version; `2' corresponds to RingCT.
	\item {\tt unlock\_time} (linha 5) - Previne que a(s) saída(s) de transacção sejam gastas antes que esse tempo passe. Trata-se aqui de uma altura de bloco, ou um carimbo no tempo de UNIX. Se o número é maior do que o tempo UNIX até á data. Este valor torna-se zero se nenhum limite é definido.  
%Prevents a transacção's outputs from being spent until the time has past. It is either a block height, or a UNIX timestamp if the number is larger than the beginning of UNIX time. It defaults to zero when no limit is specified.	
	\item {\tt vin} (linha 6-23) - lista de entradas (existem duas aqui)
%List of inputs (there are two here)
	\item {\tt amount} (linha 8) - Descontinuado (legado); campo para o montante, para transacções do tipo 1.
%Deprecated (legacy) amount field for type 1 transactions
	\item {\tt key\_offset} (linha 9) - Isto permite que os verificadores encontrem as chaves de membros de anel e compromissos na lista de blocos, o que implica óbviamente que estes membros são legítimos. O primeiro offset é absoluto dentro do histórico da lista de blocos, e cada offset subsequente é relativo ao anterior. Por exemplo, com offsets reais \{7,11,15,20\}, guarda-se na lista de blocos \{7,4,4,5\}. Os verificadores calculam o último offset como (7+4+4+5 = 20) (secção \ref{subsec:space-and-ver-rcttypefull}).
%This allows verifiers to find ring member keys and commitments in the blockchain, and makes it obvious those members are legitimate. The first offset is absolute within the blockchain history, and each subsequent offset is relative to the previous. For example, with real offsets \{7,11,15,20\}, the blockchain records \{7,4,4,5\}. Verifiers compute the last offset like (7+4+4+5 = 20) (Section \ref{subsec:space-and-ver-rcttypefull}).
	\item {\tt k\_image} (linha 12) - Imagem de chave \(\tilde{K_j}\) da secção \ref{sec:MLSAG}, em que $j = 1$ desde que esta é a primeira entrada.  
%Key image \(\tilde{K_j}\) from Section \ref{sec:MLSAG}, where $j = 1$ since this is the first input.
	\item {\tt vout} (linhas 24-35) - Lista de saídas (existem duas aqui)
%List of outputs (there are two here)
	\item {\tt amount} (linha 25) - Descontinuado (legado); campo para o montante, para transacções do tipo 1.
	\item {\tt key} (linha 27) - chave do endereço oculto de destino para a saída $t = 0$ como descrito na secção \ref{sec:one-time-addresses}.
%One-time destination key for output $t = 0$ as described in Section \ref{sec:one-time-addresses}
	\item {\tt extra} (linhas 36-39) - Diversos dados, \marginnote{src/crypto- note\_basic/ tx\_extra.h} incluíndo a {\em chave pública de transacção}, i.e. o segredo partilhado $r G$ da secção \ref{sec:one-time-addresses}, e ID de pagamento da secção \ref{sec:integrated-addresses}. Típicamente funciona da seguinte forma: cada número é um byte (de 0 a 255), e cada tipo diferente de dados têm um `tag' e um `comprimento'. O tag indica qual é a informação que se segue, e o comprimento indica quantos bytes essa informação ocupa. O primeiro número é sempre um tag. Aqui, `1' indica uma chave pública de transacção. Estas chaves têm sempre 32 bytes de comprimento, portanto não é necessário incluír o comprimento. Depois de 32 números encontra-se um novo tag : `2' o que significa `nonce extra', o seu comprimento é `9', e o próximo byte é `1' o que significa um ID de pagamento codificado (o nonce extra pode ter campos próprios por alguma razão). Depois de 8 bytes o campo extra acaba. Veja-se \cite{extra-field-stackexchange} para mais detalhes. (Na especificação original de criptonote o primeiro byte indicava o tamanho do campo. Monero não utiliza isso.) \cite{tx-extra-field}  
%Miscellaneous\marginnote{src/crypto- note\_basic/ tx\_extra.h} data, including the {\em transacção public key}, i.e. the share secret $r G$ of Section  \ref{sec:one-time-addresses}, and encrypted payment ID from Section \ref{sec:integrated-addresses}. It typically works like this: each number is one byte (it can be 0-255), and each kind of thing that can be in the field has a `tag' and `length'. The tag indicates which information comes next, and length indicates how many bytes that info occupies. The first number is always a tag. Here, `1' indicates a `transacção public key'. Tx public keys are always 32 bytes, so we don't need to include the length. Thirty-two numbers later we find a new tag `2' which means `extra nonce', its length is `9', and the next byte is `1' which means an 8-byte encrypted payment ID (the extra nonce can have fields inside it, for some reason). Eight bytes go by, and that's the end of this extra field. See \cite{extra-field-stackexchange} for more details. (note: in the original Cryptonote specification the first byte indicated the size of the field. Monero doesn't use that.) \cite{tx-extra-field}
	\item {\tt rct\_signatures} (linhas 40-50) - Primeira parte dos dados de assinatura 
%First part of signature data
	\item {\tt type} (linha 41) - Tipos de assinatura; {\tt RCTTypeBulletproof2} é tipo 4. Tipos descontinuados RingCT {\tt RCTTypeFull} e {\tt RCTTypeSimple} são tipo 1 e 2 respectivamente. Transacções de mineiro são {\tt RCTTypeNull}, tipo 0.
%Signature type; {\tt RCTTypeBulletproof2} is type 4. Deprecated RingCT types {\tt RCTTypeFull} and {\tt RCTTypeSimple} were 1 and 2 respectively. Miner transactions use {\tt RCTTypeNull}, type 0.
	\item {\tt txnFee} (linha 42) - Taxa de transacção em texto claro, neste caso 0.00003246 xmr.
%Transacção fee in clear text, in this case 0.00003246 XMR
	\item {\tt ecdhInfo} (linhas 43-47) - ‘elliptic curve diffie-hellman Info’: 
Montante oculto para cada uma das saídas $t \in \{0, ..., p-1\}$; em que $p = 2$
%Obscured amount for each of the outputs $t \in \{0, ..., p-1\}$; here $p = 2$
    \item {\tt amount} (linha 44) - O campo {\sl amount} com $t = 0$ é descrito na secção \ref{sec:pedersen_monero}.
    \item {\tt outPk} (linhas 48-49) - Compromissos para cada saída, secção \ref{sec:ringct-introduction}.
%Commitments for each output, Section \ref{sec:ringct-introduction}

    \item {\tt rctsig\_prunable} (linhas 51-132) - Segunda parte dos dados da assinatura
%Second part of signature data
    \item {\tt nbp} (linha 52) - Número de provas de domínio {\em Bulletproof} nesta transacção.
%Number of Bulletproof range proofs in this transacção
    \item {\tt bp} (linhas 53-80) - elementos da prova {\em Bulletproof}\vspace{.175cm}
%Bulletproof proof elements (Bulletproofs were not explored in this document, so we will not itemize further)
    \[\Pi_{BP} = (A, S, T_1, T_2, \tau_x, \mu, \mathbb{L}, \mathbb{R}, a, b, t)\]
    \item {\tt MGs} (linhas 81-129) - assinaturas MLSAG
%MLSAG signatures
    \item {\tt ss} (linhas 82-103) - Componentes \(r_{i,1}\) e \(r_{i,2}\) da assinatura MLSAG para a primeira entrada.\vspace{.175cm}
%from the first input's MLSAG signature\vspace{.175cm}
    \[\sigma_j(\mathfrak{m}) = (c_1, r_{1, 1}, r_{1, 2}, ..., r_{v+1, 1}, r_{v+1, 2})\]
    \item {\tt cc} (linha 104) - Componente \(c_1\) da assinatura MLSAG, para a primeira entrada.
%from aforementioned MLSAG signature
    \item {\tt pseudoOuts} (linhas 130-131) - Pseudo compromissos de saída $C'^a_j$, como descrito na secção \ref{sec:pedersen_monero}. Lembre-se que a soma destes compromissos é igual á soma dos dois compromissos de saída mais o compromisso da taxa de mineiro $f H$. 
%Pseudo output commitments $C'^a_j$, as described in Section  \ref{sec:pedersen_monero}. Please recall that the sum of these commitments will equal the sum of the two output commitments of this transacção (plus the transacção fee commitment $f H$).
\end{itemize}

\chapter{Conteúdo de bloco}
\label{appendix:block-content}

Neste apêndice mostra-se a estrutura de um bloco, nomeadamente o bloco n° 1582196. Este bloco contêm 5 transacções, e foi minerado para a lista de blocos com o carimbo no tempo 2018-05-27 21:56:01 UTC, pelo seu mineiro.
%In this appendix we show the structure of a sample block, namely the 1582196\nth block after the genesis block. The block has 5 transactions, and was added to the blockchain at timestamp 2018-05-27 21:56:01 UTC (as reported by the block's miner).

\begin{Verbatim}[commandchars=\\\{\}, numbers=left]
print_block 1582196
timestamp: 1527458161
previous hash: 30bb9b475a08f2ea6fe07a1fd591ea209a7f633a400b2673b8835a975348b0eb
nonce: 2147489363
is orphan: 0
height: 1582196
depth: 2
hash: 50c8e5e51453c2ab85ef99d817e166540b40ef5fd2ed15ebc863091ca2a04594
difficulty: 51333809600
reward: 4634817937431
\{
  "major_version": 7,
  "minor_version": 7,
  "timestamp": 1527458161,
  "prev_id": "30bb9b475a08f2ea6fe07a1fd591ea209a7f633a400b2673b8835a975348b0eb",
  "nonce": 2147489363,
  "miner_tx": \{
    "version": 2,
    "unlock_time": 1582256,
    "vin": [ \{
        "gen": \{
          "height": 1582196
        \}
      \}
    ],
    "vout": [ \{
        "amount": 4634817937431,
        "target": \{
          "key": "39abd5f1c13dc6644d1c43db68691996bb3cd4a8619a37a227667cf2bf055401"
        \}
      \}
    ],
    "extra": [ 1, 89, 148, 148, 232, 110, 49, 77, 175, 158, 102, 45, 72, 201, 193,
    18, 142, 202, 224, 47, 73, 31, 207, 236, 251, 94, 179, 190, 71, 72, 251, 110, 
    134, 2, 8, 1, 242, 62, 180, 82, 253, 252, 0
    ],
    "rct_signatures": \{
      "type": 0
    \}
  \},
  "tx_hashes": [ "e9620db41b6b4e9ee675f7bfdeb9b9774b92aca0c53219247b8f8c7aecf773ae",
                 "6d31593cd5680b849390f09d7ae70527653abb67d8e7fdca9e0154e5712591bf",
                 "329e9c0caf6c32b0b7bf60d1c03655156bf33c0b09b6a39889c2ff9a24e94a54",
                 "447c77a67adeb5dbf402183bc79201d6d6c2f65841ce95cf03621da5a6bffefc",
                 "90a698b0db89bbb0704a4ffa4179dc149f8f8d01269a85f46ccd7f0007167ee4"
  ]
\}
\end{Verbatim}



\section*{Componentes do bloco}
%Block components}

\begin{itemize}
	\item (linhas 2-10) - Informação dada pelo software cliente, isto não faz parte do bloco. 
%Block information collected by software, not actually part of the block properly speaking.
    \item {\tt is orphan} (linha 5) - Significa que este bloco foi orfão. Os {\em nodes} usualmente guardam todos os ramos durante uma situação de garfo de rede, e discartam ramos desnecessários quando uma outra dificuldade comulativa vencedora emerge, o que torna os blocos do outro garfo órfaos.
%Signifies if this block was orphaned. Nodes usually store all branches during a fork situation, and discard unnecessary branches when a cumulative difficulty winner emerges, thereby orphaning the blocks.
    \item {\tt depth} (linha 7) - A profundidade de qualquer dado bloco, é a distância desse bloco ao último bloco na lista (distância ao bloco mais recente).
%??? guarda-se isto no bloco?
%In a blockchain copy, the depth of any given block is how far back in the chain it is relative to the most recent block.
    \item {\tt hash} (linha 8) - A ID deste bloco.
%This block's block ID.
    \item {\tt difficulty} (linha 9) - A dificuldade não é guardada num bloco, desde que os utilizadores podem calcular {\em todas} as dificuldades de bloco a partir dos carimbos no tempo e com as regras na secção \ref{sec:difficulty}. 
%Difficulty isn't stored in a block, since users can compute {\em all} block difficulties from block timestamps and the rules in Section \ref{sec:difficulty}.
    \item {\tt major\_version} (linha 12) - Corresponde á versão do protocolo usada para verificar este bloco.
%Corresponds to the protocol version used to verify this block.
    \item {\tt minor\_version} (linha 13) - Originalmente destinada para ser um mecanismo entre mineiros, agora simplesmente repete a {\tt major\_version}. 
%Originally intended as a voting mechanism among miners, it now merely reiterates the {\tt major\_version}. Since the field does not occupy much space, the developers probably thought deprecating it would not be worth the effort.
    \item {\tt timestamp} (linha 14) - Uma representação com um inteiro do carimbo no tempo em UTC, posto pelo mineiro.
%An integer representation of this block's UTC timestamp, as reported by the block's miner.
    \item {\tt prev\_id} (linha 15) - A ID do bloco anterior.
%The previous block's block ID. Herein lies the essence of Monero's blockchain.
    \item {\tt nonce} (linha 16) - A nonce usada pelo mineiro do bloco para passar o alvo de dificuldade. É possível verificar que esta prova de trabalho é válida para esta nonce.  
%The nonce used by this block's miner to pass its difficulty target. Anyone can recompute the proof of work and verify the nonce is valid.
    \item {\tt miner\_tx} (linhas 17-40) - A transacção de mineiro
%This block's miner transacção.
    \item {\tt version} (linha 18) - O formato/era da transacção; version `2' corresponde a RingCT.
    \item {\tt unlock\_time} (linha 19) - A transacção de mineiro não pode ser gasta até ao bloco 1582256, até que mais 59 blocos tenham sido minerados (é um tempo de bloqueio de 60 blocos pois não pode ser gasto até que 60 intervalos de tempo entre blocos tenham passado, e.g. $2*60 = 120$ minutos).
%The miner transacção's output can't be spent until the 1582256\nth block, after 59 more blocks have been mined (it is a 60 block lock time since it can't be spent until 60 block time intervals have passed, e.g. $2*60 = 120$ minutes).
    \item {\tt vin} (linhas 20-25) - Entradas para as transacção de mineiro, não existem, pois a transacção de mineiro é usada para gerar o subsídio de bloco e coleccionar taxas de transacção.
%Inputs to the miner tx. There are none, since the miner tx is used to generate block rewards and collect transacção fees.
    \item {\tt gen} (linha 21) - Forma curta para `gerar'.
%Short for `generate'.
    \item {\tt height} (linha 22) - A altura de bloco. Cada altura de bloco só pode gerar o subsídio de bloco, uma vez.
%The block height for which this miner tx's block reward was generated. Each block height can only generate a block reward once.
    \item {\tt vout} (linhas 26-32) - Saídas da transacção de mineiro.
%Outputs of the miner tx.
    \item {\tt amount} (linha 27) - Montante da transacção de mineiro, contêm o subsídio do bloco e as taxas de transacção deste bloco. Usa a unidade atómica {\em piconero} . 
%Amount dispersed by the miner tx, containing block reward and fees from this block's transactions. Recorded in atomic units.
    \item {\tt key} (linha 29) - Endereço oculto que é proprietário do montante na transacção de mineiro.
%One-time address assigning ownership of the miner tx's amount.
    \item {\tt extra} (linhas 33-36) - Informação extra para a transacção de mineiro, inclui a chave pública de transacção.
%Extra information for the miner tx, including the transacção public key.
    \item {\tt type} (linha 38) - Tipo de transacção, neste caso `0' para {\tt RCTTypeNull}.  
%Type of transacção, in this case `0' for {\tt RCTTypeNull}, indicating a miner tx.
    \item {\tt tx\_hashes} (linhas 41-46) - Todas as ID's de transacção incluídas neste bloco (mas não a ID da transacção de mineiro, que é : \newline {\tt 06fb3e1cf889bb972774a8535208d98db164394ef2b14ecfe74814170557e6e9}).
%All transacção IDs included in this block (but not the miner tx ID, which is {\tt 06fb3e1cf889bb972774a8535208d98db164394ef2b14ecfe74814170557e6e9}).
\end{itemize}

\chapter{Bloco de génese}
%Genesis Block}
\label{appendix:genesis-block}

Neste apêndice mostra-se a estrutura do bloco génese, o bloco tem zero transacções, só envia o primeiro subsídio de bloco para thankful\_for\_today \cite{bitmonero-launched}). O fundador de Monero não adicionou um carimbo no tempo, talvez uma marca de Bytecoin, a moeda da qual provêm o código fonte de Monero. Mais sobre a história de Bytecoin pode ser lida aqui \cite{monero-history}, e aqui \cite{bytecoin-network}.  
%In this appendix we show the structure of the Monero genesis block. The block has 0 transactions (it just sends the first block reward to thankful\_for\_today \cite{bitmonero-launched}). Monero's founder did not add a timestamp, perhaps as a relic of Bytecoin, the coin Monero's code was forked from, whose creators apparently tried to hide a large pre-mine \cite{monero-history}, and may operate a shady network of cryptocurrency-related software and services \cite{bytecoin-network}.

O bloco 1 foi adicionado á rede com o carimbo no tempo 2014-04-18 10:49:53 UTC, assume-se que o bloco 0, de génese tenha sido minerado no mesmo dia. Isto corresponde com a data de lançamento anunciada por thankful\_for\_today \cite{bitmonero-launched}.
%Block 1 was added to the blockchain at timestamp 2014-04-18 10:49:53 UTC (as reported by the block's miner), so we can assume the genesis block was created the same day. This corresponds with the launch date announced by thankful\_for\_today \cite{bitmonero-launched}.

\begin{Verbatim}[commandchars=\\\{\}, numbers=left]
print_block 0
timestamp: 0
previous hash: 0000000000000000000000000000000000000000000000000000000000000000
nonce: 10000
is orphan: 0
height: 0
depth: 1580975
hash: 418015bb9ae982a1975da7d79277c2705727a56894ba0fb246adaabb1f4632e3
difficulty: 1
reward: 17592186044415
\{
  "major_version": 1,
  "minor_version": 0,
  "timestamp": 0,
  "prev_id": "0000000000000000000000000000000000000000000000000000000000000000",
  "nonce": 10000,
  "miner_tx": \{
    "version": 1,
    "unlock_time": 60,
    "vin": [ \{
        "gen": \{
          "height": 0
        \}
      \}
    ],
    "vout": [ \{
        "amount": 17592186044415,
        "target": \{
          "key": "9b2e4c0281c0b02e7c53291a94d1d0cbff8883f8024f5142ee494ffbbd088071"
        \}
      \}
    ],
    "extra": [ 1, 119, 103, 170, 252, 222, 155, 224, 13, 207, 208, 152, 113, 94, 188, 
    247, 244, 16, 218, 235, 197, 130, 253, 166, 157, 36, 162, 142, 157, 11, 200, 144, 
    209
    ],
    "signatures": [ ]
  \},
  "tx_hashes": [ ]
\}
\end{Verbatim}

\section*{Componentes do bloco de génese}
%Genesis block components}

Como foi usado o mesmo software para imprimir o bloco de génese e o bloco do apêndice \ref{appendix:block-content}, a estrutura parece ser básicamente a mesma. São dadas algumas diferenças.
%Since we used the same software to print the genesis block and the block from Appendix \ref{appendix:block-content}, the structure appears basically the same. We point out some unique differences.

\begin{itemize}
	\item {\tt difficulty} (linha 9) - A dificuldade é 1, o que significa que qualquer nonce serve.
%The genesis block's difficulty is reported as 1, which means any nonce could work.
	\item {\tt timestamp} (linha 14) - O bloco de génese não tem um carimbo no tempo significante.
%The genesis block doesn't have a meaningful timestamp.
	\item {\tt prev\_id} (linha 15) - Usam-se 32 bytes a zero para o ID anterior, por convenção.
%We use an empty 32 bytes for the previous ID, by convention.
	\item {\tt nonce} (linha 16) - $n = 10000$ por convenção.
	\item {\tt amount} (linha 27) - Isto corresponde exactamente ao cálculo para o primeiro subsídio de bloco (17.592186044415 xmr) da secção \ref{subsec:block-reward}.
%This exactly corresponds to our calculation for the first block reward (17.592186044415 XMR) in Section \ref{subsec:block-reward}.
	\item {\tt key} (linha 29) - Os primeiros moneroj foram dispersados para o fundador de Monero thankful\_for\_today.
%The very first Moneroj were dispersed to Monero's founder thankful\_for\_today.
	\item {\tt extra} (linhas 33-36) - 
Usa-se a codificação discutida no apêndice %\ref{appendix:RCTTypeBulletproof2},
o campo {\tt extra} da transacção de mineiro, só contêm uma chave pública de transacção.   
%Using the encoding discussed in Appendix \ref{appendix:RCTTypeBulletproof2}, the genesis block's miner tx {\tt extra} field just contains one transacção public key.
	\item {\tt signatures} (linha 37) - Não existem assinaturas no bloco de génese. Isto aparece aqui devido á função {\tt print\_block}. O mesmo é verdade para {\tt tx\_hashes} na linha 39.
%There are no signatures in the genesis block. This is here as an artifact of the {\tt print\_block} function. The same is true for {\tt tx\_hashes} in linha 39.
\end{itemize}

\end{appendices}
